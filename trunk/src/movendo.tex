\chapter{Movendo-se no Documento}\label{cha:Movendo-se no Documento}

A fim de facilitar o entendimento acerca das teclas e atalhos de movimentação,
faz-se útil uma breve recapitulação de conceitos relacionados. Para se entrar
em modo de inserção, estando em modo normal, pode-se pressionar qualquer uma
das teclas abaixo:

\begin{verbatim}
     i ..... entra no modo de inserção antes do caractere atual
     I ..... entra no modo de inserção no começo da linha
     a ..... entra no modo de inserção após o caractere atual
     A ..... entra no modo de inserção no final da linha
     o ..... entra no modo de inserção uma linha abaixo
     O ..... entra em modo de inserção uma linha cima
     <Esc> . sai do modo de inserção
\end{verbatim}

Uma vez no modo de inserção todas as teclas são efetivamente, assim como nos
outros editores simples, caracteres que constituem o conteúdo do texto sendo
digitado.  Para sair do modo de inserção e retornar ao modo normal digita-se
\verb+<Esc>+ ou \verb+Ctrl-[+.

As letras h, k, l, j funcionam como setas:

\begin{verbatim}
        k
     h     l
        j
\end{verbatim}

Ou seja, a letra ``k'' é usada para subir no texto, a letra ``j'' para descer,
a letra ``h'' para mover-se para a esquerda e a letra ``l'' para mover-se para
a direita. A idéia é que se consiga ir para qualquer lugar do texto sem tirar
as mãos do teclado, sendo portando alternativas para as setas de movimentação
usuais do teclado.

\section{Big words}
\label{Big words}

Para o Vim ``{\em{palavras-separadas-por-hífen}}'' são consideradas em separado, portanto se você usar,
em modo normal ``\verb+w+'' avançar entre as palavras ele pulará uma de
cada vez, no entanto se usar ``\verb+W+''
em maiúsculo (como visto) ele pulará a ``a-palavra-inteira'' :)

\begin{verbatim}
     E .... pula para o final de palavras com hifen
     B .... pula palavras com hifen (retrocede)
     W .... pula palavras hifenizadas (começo)
\end{verbatim}



Para ir para linhas específicas digite

\begin{verbatim}
     :n<Enter>  ..... vai para linha ``n''
     ngg ............ vai para linha ``n''
     nG ............. vai para linha ``n''
\end{verbatim}

onde ``\verb|n|'' corresponde ao número da linha.

Para retornar ao modo normal pressione \verb|<Esc>| ou use \verb|Ctrl-[|
(\verb|^[|).

\section{Os saltos}\label{Os saltos}

\begin{verbatim}
     gg .... vai para o início do arquivo
     G ..... vai para o final do arquivo
     0 ..... vai para o início da linha
     ^ ..... vai para o primeiro caractere da linha (ignora espaços)
     $ ..... vai para o final da linha
     yG .... copia da linha atual até o final do arquivo
     25gg .. salta para a linha 25
     '' .... salta para a linha da última posição em que o cursor estava
     fx .... para primeria ocorrência de x
     tx .... Para ir para uma letra antes de x
     Fx .... Para ir para ocorrência anterior de x
     Tx .... Para ir para uma letra após o último x
     * ..... Próxima ocorrência de palavra sob o cursor
     % ..... localiza parênteses correspondente
     `' .... salta exatamente para a posição em que o cursor estava
     d$ .... deleta do ponto atual até o final da linha
     gi .... entra em modo de inserção no ponto da última edição
     gv .... repete a última seleção visual e posiciona o cursor neste local
     gf .... abre o arquivo sob o cursor
     gd .... salta para declaração de variável sob o cursor
     gD .... salta para declaração (global) de variável sob o cursor
     w ..... move para o início da próxima palavra
     W ..... pula para próxima palavra (desconsidera hífens)
     E ..... pula para o final da próxima palavra (desconsidera hifens)
     e ..... move o cursor para o final da próxima palavra
     zt .... movo o cursor para o topo da página
     zm .... move o cursor para o meio da página
     zz .... move a página de modo com que o cursor fique no centro
     n ..... move o cursor para a próxima ocorrência da busca
     N ..... move o cursor para a ocorrência anterior da busca
\end{verbatim}

\section{Copiar e Deletar}\label{sec:Copiar e Deletar}

Deletar está associado à letra ``\verb|d|''.

\begin{verbatim}
     dd .... deleta linha atual
     D ..... deleta restante da linha
     d$ .... deleta restante da linha
     d^ .... deleta do cursor ao primeiro caractere não-nulo da linha
     d0 .... deleta do cursor ao início da linha
\end{verbatim}


``Dica'': Você pode combinar o comando de deleção ``\verb+d+'' com o
comando de movimento (considere o modo normal) para apagar até a
próxima vírgula use: ``\verb+df,+''. \\


Copiar está associado à letra ``\verb|y|''.

\begin{verbatim}
     yy .... copia a linha atual
     Y ..... copia a linha atual
     ye .... copia do cursor ao fim da palavra
     yb .... copia do começo da palavra ao cusor
\end{verbatim}

A maioria dos comandos do Vim pode ser precedida por um quantificador:

\begin{verbatim}
     5j ..... desce 5 linhas
     d5j .... deleta as próximas 5 linhas
     k ...... em modo normal sobe uma linha
     5k ..... sobe 5 linhas
     y5k .... copia 5 linhas (para cima)
     w ...... pula uma palavra para frente
     5w ..... pula 5 palavras
     d5w .... deleta 5 palavras
     b ...... retrocede uma palavra
     5b ..... retrocede 5 palavras
     fx ..... posiciona o cursor em ``x''
     dfx .... deleta até o próximo ``x''
     dgg .... deleta da linha atual até o começo do arquivo
     dG ..... deleta até o final do arquivo
     yG ..... copia até o final do arquivo
     yfx .... copia até o próximo ``x''
     y5j .... copia 5 linhas
\end{verbatim}

Podemos pular sentenças:

\begin{verbatim}
     ) .... pula uma sentença para frente
     ( .... pula uma sentença para tráz
     } .... pula um parágrafo para frente
     { .... pula um parágrafo para tráz
     y) ... copia uma sentença para frente
     d} ... deleta um parágrafo para frente
\end{verbatim}

O que foi deletado ou copiado pode ser colado:

\begin{verbatim}
     p .... cola o que foi copiado ou deletado abaixo
     P .... cola o que foi copiado ou deletado acima
     [p ... cola o que foi copiado ou deletado antes do cursor
     ]p ... cola o que foi copiado ou deletado após o cursor
\end{verbatim}

Caso tenha uma estrutura como abaixo:

\begin{verbatim}
     def pot(x):
        return x**2
\end{verbatim}

E tiver uma referência qualquer para a função \verb+pot+ e desejar
mover-se até sua definição basta posicionar o cursor sobre a palavra
\verb+pot+ e pressionar (em modo normal)

\begin{verbatim}
     gd
\end{verbatim}

Se a variável for global, ou seja, estive fora do documento
(provavelmente em outro) use:

\begin{verbatim}
     gD
\end{verbatim}

Quando definimos uma variável tipo

\begin{verbatim}
     var = `teste'
\end{verbatim}

e em algum ponto do documento houver referência a esta variável e se
desejar ver seu conteúdo fazemos

\begin{verbatim}
     [i
\end{verbatim}
Na verdade o atalho acima lhe mostrará o último ponto onde foi feita
a atribuição àquela variável que está sob o cursor, uma mão na roda
para os programadores de plantão!


Obs: observe a  barra de status do Vim se o tipo de arquivo está certo,
tipo. Para detalhes sobre como personalizar a barra de status na seção
\ref{Função para barra de status}.

\begin{verbatim}
     ft=python
\end{verbatim}

a busca por definições de função só funciona se o tipo de arquivo
estiver correto

\begin{verbatim}
     :set ft=python
\end{verbatim}

outro detalhe para voltar ao último ponto em que você estava

\begin{verbatim}
     ''
\end{verbatim}

\section{Paginando}
\label{Paginando}

Para rolar uma página de cada vez (em modo normal)

\begin{verbatim}
     Ctrl-f
     Ctrl-b
\end{verbatim}


\begin{verbatim}
     :h jumps .... ajuda sobre a lista de saltos
     :jumps ...... exibe a lista de saltos
     Ctrl-i ... salta para a posição mais recente
     Ctrl-o ... salta para a posição mais antiga
     '0 ....... abre o último arquivo editado
     '1 ....... abre o penúltimo arquivo editado
     gd ....... pula para a difinição de uma variável
     } ........ pula para o fim do parágrafo
     10| ...... pula para a coluna 10
     [i ....... pula para definição de variável sob o cursor
\end{verbatim}

Observação: lembre-se

\begin{verbatim}
     ^ .... equivale a Ctrl
     ^I ... equivale a Ctrl-I
\end{verbatim}

Você pode abrir vários arquivos tipo \verb|vim *.txt| e fazer
algo como gravar e ir para o próximo arquivo com o comando a
seguir:

\begin{verbatim}
     :wn
\end{verbatim}

Ou gravar um arquivo e voltar ao anterior

\begin{verbatim}
     :wp
\end{verbatim}

Pode ainda ``rebobinar'' sua lista de arquivos :)

\begin{verbatim}
     :rew[wind]
\end{verbatim}

Ou ir para o primeiro

\begin{verbatim}
     :fir[ist]
\end{verbatim}

\section{Usando marcadores}
\label{Usando marcadores}

No Vim podemos marcar o ponto em que o cursor está, você deve estar em
modo normal, portanto pressione

\begin{verbatim}
     <Esc>
\end{verbatim}

você estará em modo normal, assim podem pressionar a tecla ``\verb+m+''
seguida de uma das letras do alfabeto

\begin{verbatim}
     ma ....... cria uma marca `a'
     `a ....... move o cursor para a marca `a'
\end{verbatim}

\section{Marcas globais}
\label{Marcas globais}
Marcas globais são marcas que permitem pular de um arquivo a outro.
Para criar uma marca global use a letra que designa a marca em
maiúsculo.

\begin{verbatim}
     mA ....... cria uma marca global A
\end{verbatim}

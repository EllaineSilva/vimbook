%%%%%%%%%%%%%%%%%%%%%%%%%%%%%%%%%%%%%%%%%%%%%%%%%%%%%%%%%%%%%%%%%%%%%%%%
% vim:enc=utf-8:ts=5:sw=5:et
%%%%%%%%%%%%%%%%%%%%%%%%%%%%%%%%%%%%%%%%%%%%%%%%%%%%%%%%%%%%%%%%%%%%%%%%

\chapter{Comandos Externos}
O Vim permite executar comandos externos para processar ou filtrar o
conteúdo de um arquivo. De forma geral, fazemos isso digitando (no
modo normal):

\begin{verbatim}
     :!ls .... visualiza o conteúdo do diretório
\end{verbatim}

Lembrando que anexando um simples ponto, a saída do comando torna-se o 
documento que está sendo editado:

\begin{verbatim}
     :.!ls .... imprime na tela o conteúdo do diretório
\end{verbatim}

A seguir, veja alguns exemplos de utilização:

\section{Ordenando}
Podemos usar o comando {\em sort} que ordena o conteúdo de um arquivo dessa forma:

\begin{verbatim}
     :5,15!sort ..... odena da linha 5 até a linha 15
\end{verbatim}

O comando acima ordena da linha 5 até a linha 15.

O comando {\em sort} existe tanto no Windows quanto nos sistemas Unix.
Digitando simplesmente {\em sort}, sem argumentos, o comportamento padrão
é de classificar na ordem alfabética (baseando-se na linha inteira).
Para maiores informações sobre argumentos do comando {\em sort}, digite

\begin{verbatim}
     sort --help ou man sort (no Unix) ou
     sort /? (no Windows).
\end{verbatim}

\section{Removendo linhas duplicadas}

\begin{verbatim}
     :%!uniq
\end{verbatim}

O caractere ``\%'' representa a região equivalente ao arquivo atual inteiro.
A versão do Vim 7 em diante tem um comando {\em sort} que permite remover
linhas duplicadas {\em uniq} e ordenar, sem a necessidade de usar comandos
externos, para mais detalhes:

\begin{verbatim}
     :h sort
\end{verbatim}

\section{Ordenando e removendo linhas duplicadas no Vim 7}

\begin{verbatim}
     :sort u
\end{verbatim}

Quando a ordenação envolver números faça:

\begin{verbatim}
     :sort n
\end{verbatim}

\section{{\em Beautifiers}}

A maior parte das linguagens de programação possui ferramentas
externas chamadas {\em beautifiers}, que servem para embelezar o código,
através da indentação e espaçamento. Por exemplo, para embelezar um
arquivo HTML é possível usar a ferramenta ``tidy'', do W3C:

\begin{verbatim}
     :%!tidy
\end{verbatim}

\section{Compilando e verificando erros}
Se o seu projeto já possui um {\tt Makefile}, então você pode fazer uso do comando
{\tt :make} para poder compilar seus programas no conforto de seu Vim:

\begin{verbatim}
     :make
\end{verbatim}

A vantagem de fazer isso é poder usar outra ferramenta bastante interessante, a janela
de {\em quickfix}:

\begin{verbatim}
     :cwindow
\end{verbatim}

O comando {\tt cwindow} abrirá uma janela em um {\em split} horizontal com a
listagem de erros e {\em warnings}.  Você poderá navegar pela lista usando os
cursores e ir diretamente para o arquivo e linha da ocorrência.

Modificando o compilador, o comando {\tt make} pode mudar sua ação.

\begin{verbatim}
	:compiler javac
	:compiler gcc
	:compiler php
\end{verbatim}

Note que {\em php} não tem um compilador. Logo, quando executado, o {\tt make} irá verificar
por erros de sintaxes.

\begin{verbatim}
	:compiler
\end{verbatim}

O comando acima lista todos os compiladores suportados.

\section{Grep}

Do mesmo jeito que você usa {\tt grep} na sua linha de comando você pode usar
o {\tt grep} interno do Vim. Exatamente do mesmo jeito:

\begin{verbatim}
     :grep <caminho> <padrão> <opções>
\end{verbatim}

Use a janela de {\em quickfix} aqui também para exibir os resultados do {\tt
grep} e poder ir diretamente a eles.

\section{Referências}
* \url{http://www.dicas-l.com.br/dicas-l/20070119.php}

%%%%%%%%%%%%%%%%%%%%%%%%%%%%%%%%%%%%%%%%%%%%%%%%%%%%%%%%%%%%%%%%%%%%%%%%%%%%%%%%

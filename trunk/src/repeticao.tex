%%%%%%%%%%%%%%%%%%%%%%%%%%%%%%%%%%%%%%%%%%%%%%%%%%%%%%%%%%%%%%%%%%%%%%%%
% vim:enc=utf-8:ts=5:sw=5:et
%%%%%%%%%%%%%%%%%%%%%%%%%%%%%%%%%%%%%%%%%%%%%%%%%%%%%%%%%%%%%%%%%%%%%%%%

\chapter{Repetição de Comandos}\label{Repetição de comandos}

Para repetir a última edição saia do modo de Inserção e pressione ponto (.):

\begin{verbatim}
     .
\end{verbatim}

Para inserir um texto que deve ser repetido várias vezes:

\begin{verbatim}
     # Posicione o cursor no local desejado;
     # Digite o número de repetições;
     # Entre em modo de inserção;
     # Digite o texto;
     # Saia do modo de inserção (tecle Esc).
\end{verbatim}

Por exemplo, se você quiser inserir oitenta traços numa linha, em vez
de digitar um por um, você pode digitar o comando:

\begin{verbatim}
     80i-<Esc>
\end{verbatim}

Veja, passo a passo, o que aconteceu:

 Antes de entrar em modo de inserção usamos um quantificador

\begin{verbatim}
     `80'
\end{verbatim}

 depois iniciamos o modo de inserção

\begin{verbatim}
     i
\end{verbatim}

depois digitamos o caractere a ser repetido

\begin{verbatim}
     -
\end{verbatim}

e por fim saímos do modo de inserção

\begin{verbatim}
     <Esc>
\end{verbatim}

Se desejássemos digitar 10 linhas com o texto

\begin{verbatim}
     isto é um teste
\end{verbatim}

deveríamos então fazer assim:
   
\begin{verbatim}
     <Esc> .. para ter certeza que ainda estamos no modo normal
     10 ..... quantificador antes
     i ...... entrar no modo de inserção
     isto é um teste <Enter>
     <Esc> .. voltar ao modo normal
\end{verbatim}

\section{Repetindo a digitação de uma linha }

\begin{verbatim}
     modo de inserção
     Ctrl-y .......... repete a linha acima 
     Ctrl-e .......... repetira linha abaixo 
     Ctrl-x Ctrl-l ... repete linhas completas
\end{verbatim}


O atalho {\tt Ctrl-x Ctrl-l} só funcionará para uma 
linha semelhante, experimente digitar:

\begin{verbatim}
     uma linha qualquer com algum conteúdo
     uma linha <Ctrl-x Ctrl-l>
\end{verbatim}

e veja o resultado

\section{Guardando trechos em ``registros''}
\label{sec:Guardando trechos em ``registros''}

Os registradores ``z'' são uma espécie de área de transferência múltipla.

Você deve estar em modo normal e então digitar uma aspa dupla e uma
das 26 letras do alfabeto, em seguida uma ação por exemplo, `y'
(copiar) `d' (apagar). Depois, mova o cursor para a linha
desejada e cole com "rp, onde `r' corresponde ao
registrador para onde o trecho foi copiado.

\begin{verbatim}
     "ayy ... copia a linha atual para o registrador ``a''
     "ap  ... cola o conteúdo do registrador ``a'' abaixo
     "bdd ... apaga a linha atual para o registrador ``b''
\end{verbatim}

\section{Macros: gravando comandos}\label{Macros: gravando comandos}

Imagine que você tem o seguinte trecho de código:

\begin{verbatim}
     stdio.h
     fcntl.h
     unistd.h
     stdlib.h
\end{verbatim}

e quer que ele fique assim:

\begin{verbatim}
     #include "stdio.h"
     #include "fcntl.h"
     #include "unistd.h"
     #include "stdlib.h"
\end{verbatim}

Não podemos simplesmente executar repetidas vezes um comando do Vim, pois
precisamos incluir texto tanto no começo quanto no fim da linha?  É necessário
mais de um comando para isso.  É aí que entram as macros. Podemos gravar até 26
macros, já que elas são guardadas nos registros do Vim, que são identificados
pelas letras do alfabeto. Para começar a gravar uma macro no registro ``a'',
digitamos

\begin{verbatim}
     qa
\end{verbatim}

No modo Normal. Tudo o que for digitado a partir daí será gravado no
registro ``a'' até que terminemos com o comando
\verb|<Esc>q| novamente (no modo Normal). Assim,
podemos solucionar nosso problema:

\begin{verbatim}
     <Esc> ....... para garantir que estamos no modo normal
     qa .......... inicia a gravação da macro `a'
     I ........... entra no modo de inserção no começo da linha
     #include " .. insere #include "
     <Esc> ....... sai do modo de inserção
     A" .......... insere o último caractere
     <Esc> ....... sai do modo de inserção
     j ........... desce uma linha
     <Esc> ....... sai do modo de inserção
     q ........... para a gravação da macro
\end{verbatim}

Agora você só precisa posicionar o cursor na primeira letra de uma linha como esta

\begin{verbatim}
     stdio.h
\end{verbatim}

E executar a macro do registro ``a'' quantas vezes for necessário,
usando o comando \verb|@a|. Para executar quatro vezes, digite:

\begin{verbatim}
     4@a
\end{verbatim}

Este comando executa quatro vezes o conteúdo do registro ``a''.

Caso tenha executado a macro uma vez pode repeti-la com o comando

\begin{verbatim}
     @@
\end{verbatim}

\section{Repetindo substituições }
Se você fizer uma substituição em um intervalo como abaixo

\begin{verbatim}
     :5,32s/isto/aquilo/g
\end{verbatim}

Pode repetir esta substituição em qualquer linha que estiver apenas usando este símbolo

\begin{verbatim}
     &
\end{verbatim}

O Vim substituirá na linha corrente ``isto'' por ``aquilo''. Podemos
repetir a última substituição globalmente assim:
   
\begin{verbatim}
     g&
\end{verbatim}

\section{Repetindo comandos}\label{Repetindo comandos}

\begin{verbatim}
     @:
\end{verbatim}

O atalho acima repete o último comando no próprio modo de comandos

\section{{\em Scripts} Vim}\label{Scripts Vim}
Usando um {\em script} para modificar um nome em vários arquivos: 
Crie um arquivo chamado {\tt subst.vim} contendo os comandos de substituição e o
comando de salvamento {\tt :wq}.

\begin{verbatim}
     %s/bgcolor="e"/bgcolor="e"/g
     wq
\end{verbatim}

Para executar um {\em script}, digite o comando

\begin{verbatim}
     :source nome_do_script.vim
\end{verbatim}

O comando ``{\tt :source}'' também pode ser abrevidado com ``{\tt :so}''
bem como ser usado para testar um esquema de cor:

\begin{verbatim}
    :so tema.vim
\end{verbatim}

\section{Usando o comando {\tt bufdo}}\label{Usando o comando bufdo}

Com o comando {\tt :bufdo} podemos executar um comando em um
conjunto de arquivos de forma rápida. No exemplo a seguir, abriremos
todos os arquivos HTML do diretório atual, efetuarei uma substituição
e em seguida salvo todos.

\begin{verbatim}
     vim *.html
     :bufdo %s/bgcolor="e"/bgcolor="e"/g | :wall
\end{verbatim}

Para fechar todos os arquivos faça:

\begin{verbatim}
     :qall
\end{verbatim}

O comando {\tt :wall} salva ``{\em write}'' todos ``{\em all}'' os arquivos
abertos pelo comando {\tt vim *.html}. Opcionalmente você pode
combinar ``{\tt :wall}'' e ``{\tt :qall}'' com o comando {\tt :wqall}, que
salva todos os arquivos abertos e em seguida sai do Vim.

\section{Colocando a última busca em um comando }
Observação: lembre-se \verb|Ctrl = ^|

\begin{verbatim}
     :^r/
\end{verbatim}

\section{Inserindo o nome do arquivo no comando }

\begin{verbatim}
     :^r%
\end{verbatim}

\section{Inserindo o último comando }

\begin{verbatim}
     ^r:
\end{verbatim}

Se preceder com ``:'' você repete o comando, equivale a acessar o histórico de
comandos com as setas

\begin{verbatim}
     :^r:
\end{verbatim}

\section{Para repetir exatamente a última inserção }

\begin{verbatim}
     i<c-a>
\end{verbatim}

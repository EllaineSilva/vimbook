\chapter{Plugins}\label{Plugins}

``Plugins'' são um meio de estender as funcionalidades do Vim, há
``plugins'' para diversas tarefas, desde wikis para o Vim até
ferramentas de auxílio a navegação em arquivos com é o caso do
``plugin'' \url{http://www.vim.org/scripts/script.php?script\_id=1658}
NerdTree, que divide uma janela que permite navegar pelos diretórios
do sistema a fim de abrir arquivos a serem editados.

\section{Como testar um plugin sem instalá-lo?}
\label{Como testar um plugin sem instala-lo?}

\begin{verbatim}
     :source <path>/<plugin>
\end{verbatim}

Caso o plugin atenda suas necessidades você pode instala-lo. Este
procedimento também funciona para temas de cor!



No GNU/Linux
\begin{verbatim}
     ~/.vim/plugin/
\end{verbatim}

No Windows

\begin{verbatim}
     ~/vimfiles/plugin/
\end{verbatim}

Obs: Caso não exista a pasta você pode criá-la!

Exemplo no GNU/Linux

\begin{verbatim}
     + /HOME/USER
           |
           |
            + .VIM
                |
                |
                + PLUGIN
\end{verbatim}

Obs: Alguns plugins dependem da versão do Vim, para saber qual
a que está atualmente instalada:

\begin{verbatim}
     :ver
\end{verbatim}

\section{Plugin para \LaTeX}
\label{Plugin para LaTeX}
Um plugin completo para \LaTeX está acessível aqui: \url{http://vim-latex.sourceforge.net/}
Uma vez adicionado o plugin você pode inserir seus {\em templates}
em:

\begin{verbatim}
     ~/.vim/ftplugin/latex-suite/templates
\end{verbatim}


\section{Criando folders para arquivos \LaTeX}
\label{Criando folders para arquivos LaTeX}

\begin{verbatim}
     set foldmarker=\\begin,\\end
     set foldmethod=marker
\end{verbatim}

Adicionar marcadores ({\em labels}) às seções de um documento \LaTeX
\begin{verbatim}
     :.s/^\(\\section\)\({.*}\)/\1\2\r\\label\2
     
     : ........... comando
     / ........... inicia padrão de busca
     ^ ........... começo de linha
     \(palavra\) . agrupa um trecho
     \(\\section\) agrupa `\section'
     \\ .......... torna \ literal
     { ........... chave literal
     .* .......... qualquer caractere em qualquer quantidade
     } ........... chave literal
     / ........... finaliza parão de busca
     \1 .......... repeter o grupo 1 \(\\section\) 
     \2 .......... repete o grupo 2 \({.*\}\)
	 \r .......... insere quebra de linha
	 \\ .......... insere uma barra invertida
     \2 .......... repete o nome da seção
\end{verbatim}

\section{Criando seções \LaTeX}\label{Criando seções latex}
o comando abaixo substitui

\begin{verbatim}
     ==seção==
\end{verbatim}

   por

\begin{verbatim}
     \section{seção}
\end{verbatim}

\begin{verbatim}
     :.s/^==\s\?\([^=]*\)\s\?==/\\section{\1}/g
     
     : ......... comando
     . ......... linha atual
     s ......... substitua
     ^ ......... começo de linha
     == ........ dois sinais de igual
     \s\? ...... seguido ou não de espaço
     [^=] ...... não pode haver = (^ dentro de [] é negação)
     * ......... diz que o que vem antes pode vir zero ou mais vezes
     \s\? ...... seguido ou não de espaço
     \\ ........ insere uma barra invertida
     \1 ........ repete o primeiro trecho entre ()
\end{verbatim}

\section{Plugin para manipular arquivos}
\url{http://www.vim.org/scripts/script.php?script_id=2337#0.1.9}
Para entender este plugin acesse este vídeo:
 \url{http://www.screencast.com/t/P6nJkJ0DE}


\section{Complementação de códigos}
\label{Complementação de códigos}

O ``plugin'' snippetsEmu é um misto entre complementação de códigos e
os chamados modelos ou {\em templates}. Insere um trecho de código pronto,
mas vai além disso, permitindo saltar para trechos do modelo inserido
através de um atalho configurável de modo a agilizar o trabalho do
programador. \url{http://www.vim.org/scripts/script.php?script\_id=1318}

\section{Instalação}
\label{Instalação}

Um artigo ensinando como instalar o ``plugin'' snippetsEmu pode ser lido aqui:
 \url{http://vivaotux.blogspot.com/2008/03/instalando-o-plugin-snippetsemu-no-vim.html}

\section{Um wiki para o Vim}
\label{Um wiki para o Vim}

O ``plugin'' wikipot implementa um wiki para o Vim no qual você define
um ``link'' com a notação WikiWord, onde um ``link'' é uma palavra que
começa com uma letra maiúscula e tem outra letra maiúscula no meio
Obtendo o plugin neste link: \url{http://www.vim.org/scripts/script.php?script\_id=1018}.

\section{Acessando documentação do python no Vim}\label{Acessando documentação do python no Vim}

 \url{http://www.vim.org/scripts/script.php?script\_id=910}

\section{Formatando textos planos com syntax}
\label{Formatando textos planos com syntax}
\url{http://www.vim.org/scripts/script.php?script\_id=2208&rating=helpful#1.3}

Veja como instalar o este plugin no capítulo \ref{cha:Um wiki para o Vim}.

\section{Movimentando em camel case}
\label{Movimentando em camel case}

O {\em plugin} \href{http://www.vim.org/scripts/script.php?script_id=1905}{{\tt CamelCaseMotion}}
auxilia a navegação em palavras em \href{http://en.wikipedia.org/wiki/Camel_case}{{\em camel case}}
ou separadas por sublinhados, através de mapeamentos similares aos que fazem a movimentação normal entre 
strings, e é um recurso de grande ajuda quando o editor é utilizado para programação. 

Após instalado o plugin, os seguintes atalhos ficam disponíveis:
\begin{description}
	\item [,w] Movimenta para a próxima posição {\em camel} dentro da string
	\item [,b] Movimenta para a posição {\em camel} anterior dentro da string
	\item [,e] Movimenta para o caracter anterior à proxima posição {\em camel} dentro da string
\end{description}

Fonte: \url{http://eustaquiorangel.com/posts/522}

\section{Plugin FuzzyFinder}
\label{sec:Plugin FuzzyFinder}
                                                                       
Este plugin é a implementação de um recurso do editor 
Texmate\footnote{Editor de textos da Apple com muitos recursos}.
Sua proposta é acessar de forma rápida:


Para ver o plugin em ação acesse este link: 
\url{http://vimeo.com/2938498}.
	
\begin{enumerate}
\item Arquivos \verb|:FuzzyFinderFile|
\item Arquivos recem editados \verb|:FuzzyFinderMruFile|
\item Comandos recem utilizados \verb|:FuzzyFinderMruCmd|
\item Favoritos \verb|:FuzzyFinderAddBookmark, :FuzzyFinderBookmarks|
\item Navegação por diretórios \verb|:FuzzyFinderDir|
\item Tags {\tt :FuzzyFinderTag}
\end{enumerate}

\subsection{Obtendo e instalando o FuzzyFinder}
O plugin pode ser obtido no seguinte endereço: 
\url{http://www.vim.org/scripts/script.php?script_id=1984},
para instala-lo basta copiar para o diretorio 
{\tt ~/.vim/plugin}.




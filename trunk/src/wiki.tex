\chapter{Um Wiki para o Vim}\label{cha:Um Wiki para o Vim}

É inegável a facilidade que um wiki nos traz, os documentos são
indexados e linkados de forma simples. Já pesquisei uma porção de
wikis e, para uso pessoal recomendo o potwiki.  O ``linkk'' do potwiki é:
\url{http://www.vim.org/scripts/script.php?script_id=1018}.
O potwiki é um wiki completo para o Vim, funciona localmente embora
possa ser aberto remotamente via ssh\footnote{Sistema de acesso remoto}.
Para criar um ``link'' no potwiki basta usar WikiNames, são nomes
iniciados com letra maiúscula e que contenham outra letra em maiúsculo
no meio. \\


Ao baixar o arquivo salve em \verb|~/.vim/plugin|. \\



Mais ou menos na linha 53 do potwiki \verb|~/.vim/plugin/potwiki.vim| você
define onde ele guardará os arquivos, no meu caso
\verb|/home/docs/textos/wiki|. a linha ficou assim:

\begin{verbatim}
     call s:default('home',"~/.wiki/HomePage")
\end{verbatim}

Outra forma de indicar a página inicial seria colocar no seu .virmc

\begin{verbatim}
     let potwiki_home = "$HOME/.wiki/HomePage"
\end{verbatim}

\section{Como usar}
\label{Como usar}

O potwiki trabalha com WikiWords, ou seja, palavras iniciadas com
letras em maiúsculo e que tenham outra letra em maiúsculo no meio (sem
espaços) para iniciá-lo abra o Vim e pressione \verb|\ww|.

\begin{verbatim}
     <Leader> é igual a \   - veja :help leader
     \ww  .... abra a sua HomePage
     \wi  .... abre o Wiki index
     \wf  .... segue uma WikiWords (can be used in any buffer!)
     \we  .... edite um arquivo Wiki
     \\   .... Fecha o arquivo
     <CR> .... segue WikiWords embaixo do cursor <CR> é igual a Enter
     <Tab>.... move para a próxima WikiWords
     <BS> .... move para os WikiWords anteriores (mesma página)
     \wr  .... recarrega WikiWords
\end{verbatim}

\section{Salvamento automático para o Wiki }
\label{Salvamento automático para o wiki }
Procure por uma seção {\em autowrite} no manual do potwiki

\begin{verbatim}
     :help potwiki
\end{verbatim}

O valor que está em zero deverá ficar em 1

\begin{verbatim}
     call s:default(`autowrite',0)
\end{verbatim}

\section{Dicas}
\label{Dicas}
Como eu mantenho o meu wiki oculto ``.wiki'' criei um ``link'' para a pasta de textos

\begin{verbatim}
     ln -s ~/.wiki /home/sergio/docs/textos/wiki
\end{verbatim}

Vez por outra entro na pasta \verb|~/docs/textos/wiki| e crio um
pacote tar.gz e mando para ``web'' como forma de manter um ``backup''.

\section{Problemas com codificação de caracteres}
\label{Problemas com codificação de caracteres}

Atualmente uso o Ubuntu em casa e ele já usa utf-8. Ao restaurar meu
``backup'' do wiki no Kurumin os caracteres ficaram meio estranhos,
daí fiz:

\begin{verbatim}
     baixei o pacote [recode]
     # apt-get install recode
     
     para recodificar caracteres de utf-8 para isso faça:
     recode -d u8..l1 arquivo
\end{verbatim}

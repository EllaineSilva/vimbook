%%%%%%%%%%%%%%%%%%%%%%%%%%%%%%%%%%%%%%%%%%%%%%%%%%%%%%%%%%%%%%%%%%%%%%%%
% vim:enc=utf-8:ts=5:sw=5:et
%%%%%%%%%%%%%%%%%%%%%%%%%%%%%%%%%%%%%%%%%%%%%%%%%%%%%%%%%%%%%%%%%%%%%%%%

\chapter{Salvando Sessões de Trabalho}
\label{cha:Salvando Sessões de Trabalho}

Suponha a situação em que um usuário está trabalhando em um projeto no qual
vários arquivos são editados simultaneamente; quatro arquivos estão abertos,
algumas macros foram criadas e variáveis que não constam no \verb|vimrc| foram
definidas. Em uma situação normal, se o Vim for fechado a quase totalidade
dessas informações se perde\footnote{Algumas informações, no entanto, são
automaticamente armazenadas no arquivo {\tt viminfo}; veja {\tt :h viminfo} }; 
para evitar isto uma sessão pode ser criada, gerando-se um
``retrato do estado atual'', e então restaurada futuramente pelo
usuário---na prática é como se o usuário não tivesse saído do editor. 

Uma sessão do Vim guarda, portanto, uma série de informações sobre a edição
corrente, de modo a permitir que o usuário possa restaurá-la quando desejar.
Sessões são bastante úteis, por exemplo, para se alternar entre diferentes
projetos, carregando-se rapidamente os arquivos e definições relativas a cada
projeto.

\section{O que uma sessão armazena?}

Uma sessão é composta das seguintes informações:

\begin{itemize}
 \item Mapeamentos globais
 \item Variáveis globais
 \item Arquivos abertos incluindo a lista de {\it buffers}
 \item Diretório corrente ({\tt :h curdir})
 \item Posição e tamanho das janelas (o {\em layout}) 
\end{itemize}

\section{Criando sessões}
\label{sec:Criando sessões}

Sessões são criadas através do comando \verb|:mksession|:

\begin{verbatim}
     :mksession sessao.vim .... cria a sessão e armazena-a em sessao.vim
     :mksession! sessao.vim ... salva a sessão e sobrescreve-a em sessao.vim
\end{verbatim}

\section{Restaurando sessões}

Após gravar sessões, elas podem ser carregadas ao iniciar o Vim:

\begin{verbatim}
     vim -S sessao.vim
\end{verbatim}

ou então de dentro do próprio Vim (no modo de comando):

\begin{verbatim}
     :so sessao.vim
\end{verbatim}

Após restaurar a sessão, o nome da sessão corrente é acessível através de uma
variável interna ``\verb+v:this_session+''; caso queira-se exibir na linha de
comando o nome da sessão ativa (incluindo o caminho), faz-se:

\begin{verbatim}
     :echo v:this_session
\end{verbatim}

Podemos fazer mapeamentos para atualizar a sessão atual e exibir detalhes da
mesma:

\begin{verbatim}
     "mapeamento para gravar sessão
     nmap <F4> :wa<Bar>exe "mksession! " . v:this_session<CR>:so ~/sessions/

     "mapeamento para exibir a sessão ativa
     map <s-F4> <esc>:echo v:this_session<cr>
\end{verbatim}

\section{{\tt Viminfo}}\label{sec:Viminfo}

Se o Vim for fechado e iniciado novamente, normalmente perderá uma porção
considerável de informações. A diretiva {\tt viminfo} pode ser usada para
memorizar estas informações.

\begin{itemize}
\item Histórico da linha de comando
\item Histórico de buscas
\item Histórico de entradas {\em input-line history}
\item Conteúdo de registros não vazios
\item Marcas de vários arquivos
\item Último padrão de busca/substituição 
\item A lista de {\em buffers}
\item Variáveis globais
\end{itemize}

Deve-se colocar no arquivo de configuração algo como:
 
\begin{verbatim}
set viminfo=%,'50,\"100,/100,:100,n
\end{verbatim}

Algumas opões da diretiva {\tt viminfo}:

\begin{description}
\item [!] Quando incluído salva e restaura variáveis globais (variáveis
com letra maiúscula) e que não contém letras em minúsculo como MATENHAISTO.

\item ["] Número máximo de linhas salvas para cada registro.

\item [\%] Quando incluído salva e restaura a lista de {\em buffers}. Caso o Vim seja 
iniciado com um nome como argumento, a lista de {\em buffers} não é restaurada. {\em Buffers} sem
nome e {\em buffers} de ajuda não são armazenados no {\tt viminfo}.
                                      
\item ['] Número máximo de arquivos recém editados.

\item [/] Máximo de itens do histórico de buscas.

\item [:] Máximo de itens do histórico da linha de comando

\item [<] Número máximo de linhas salvas por cada registro, se zero
os registros não serão salvos. Quando não incluído, todas as linhas são salvas.

\end{description}

Para ver mais opções sobre o arquivo `{\tt viminfo}' leia `{\tt :h viminfo}'.
Pode-se também usar um arquivo de ``Sessão''. A diferença é que `{\tt viminfo}' não 
depende do local de trabalho (escopo).  
Quando o arquivo `{\tt viminfo}' existe e não está vazio, as informações novas
são combinadas com as existentes. A opção `{\tt viminfo}' é uma string contendo
informações sobre o que deve ser armazenado, e contém limites de o quanto 
vai ser armazenado para cada item.

%%%%%%%%%%%%%%%%%%%%%%%%%%%%%%%%%%%%%%%%%%%%%%%%%%%%%%%%%%%%%%%%%%%%%%%%%%%%%%%%

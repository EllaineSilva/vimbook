\chapter{Registros}
\label{Registros}

O Vim possui nove tipos de registros, cada tipo tem uma utilidade
específica, por exemplo você pode usar um registro que guarda o último
comando digitado, pode ainda imprimir dentro do texto o nome do
próprio arquivo, vamos aos detalhes.

\begin{itemize}
   \item O registro sem nome ``''
   \item 10 registros nomeados de ``9''
   \item O registro de pequenas deleções "-
   \item 26 registros nomeados de ``z'' ou de ``Z''
   \item 4 registros somente leitura
   \item O registro de expressões "=
   \item Os registro de seleção e  "*, "+ and "~
   \item O registro ``o''
   \item Registro do último padrão de busca "/
\end{itemize}

\section{O registro sem nome ``''}
\label{O registro sem nome ``''}

Armazena o conteúdo de ações como:

\begin{verbatim}
     d ....... deleção
     s ....... substituição
     c ....... um outro tipo de modificação
     x ....... apaga um caractere
     yy ...... copia uma linha inteira
\end{verbatim}

Para acessar o conteúdo deste registro basta usar as letras ``{\tt p}'' ou ``{\tt P}'' que
na verdade são comandos para colar abaixo da linha atual e acima da
linha atual (em modo normal)

\section{Registros nomeados de 0 a 9}
\label{Registros nomeados de 0 a 9}

O registro zero armazena o conteúdo da última cópia `yy', à partir do
registro 1 vão sendo armazenadas as deleções sucessivas de modo que a
mais recente deleção será armazenada no registro 1 e os registros vão
sendo incrementados em direção ao nono.  Deleção menores que uma linha
não são armazenadas nestes registros, caso em que o Vim usa o registro
de pequenas deleções ou que se tenha especificado algum outro
registro.

\begin{verbatim}
     :help registers
\end{verbatim}

\section{Registro de pequenas deleções}
\label{Registro de pequenas deleções}
Quando você {\em deleta} algo menor que uma linha o Vim armazena os dados deletados neste registro.

\section{Registros nomeados de ``a até z'' ou ``A até Z''}
\label{Registros nomeados de ``a até z'' ou ``A até Z''}
Você pode armazenar uma linha em modo normal assim:

\begin{verbatim}
     "ayy
\end{verbatim}

Desse modo você guardou o conteúdo da linha no registro ``a'' caso
queira armazenar mais uma linha no registro ``a'' use este comando

\begin{verbatim}
     "Add
\end{verbatim}

Neste outro caso apaguei a linha corrente adicionando-a ao final do registro ``a''.

\begin{verbatim}
     "ayip .. copia o parágrafo atual para o registro ``a''
     "a ..... registro a
     y ...... yank (copia)
     ip ..... inner paragraph (este parágrafo)
\end{verbatim}

\section{Registros somente leitura ``: . \% \#''}
\label{Registros somente leitura}

\begin{verbatim}
     ": ..... armazena o último comando
     ". ..... armazena uma cópia do último texto inserido
     "% ..... contém o nome do arquivo corrente
     "# ..... contém o nome do arquivo alternativo
\end{verbatim}

Uma forma prática de usar registros em modo de inserção é usando
\verb|Ctrl-r|


\begin{verbatim}
     Ctrl-r % .... insere o nome do arquivo atual
     Ctrl-r : .... insere o último comando digitado
     Ctrl-r / .... insere a última busca efetuada
     Ctrl-r a .... insere o registro `a'
\end{verbatim}

Em modo de inserção você pode repetir a última inserção de texto
simplesmente pressionando

\begin{verbatim}
     Ctrl-a
\end{verbatim}

\section{Registro de expressões}
\label{Registro de expressões}

\begin{verbatim}
     "=
\end{verbatim}

Este registro na verdade é usado em algumas funções avançadas.

\section{Registros de arrastar e mover}
\label{Registros de arrastar e mover}

O registro 
\begin{verbatim}
     "*
\end{verbatim}
 é responsável por armazenar o último texto selecionado (p.e., através do
mouse). Já o registro 
\begin{verbatim}
     "+
\end{verbatim}
é o denominado ``área de transferência'', normalmente utilizado para se
transferir conteúdos entre aplicações---este registro é preenchido, por
exemplo, usando-se a típica combinação {\tt Ctrl-v} encontrada em muitas
aplicações. Finalmente, o registro 
\begin{verbatim}
     "~
\end{verbatim}
armazena o texto colado pela operação mais recente de ``arrastar-e-soltar''
({\em drag-and-drop}). 

\section{Registro buraco negro "\_}
\label{Registro buraco negro}
Use este registro quando não quiser alterar os demais registros, por exemplo: se você deletar a linha atual,

\begin{verbatim}
     dd
\end{verbatim}

Esta ação irá colocar a linha atual no registro numerado 1, caso não
queira alterar o conteúdo do registro 1 apague para o buraco negro
assim:

\begin{verbatim}
     "_dd
\end{verbatim}

\section{Registros de buscas ``/''}
\label{Registros de buscas ``/''}

Se desejar inserir em uma substituição uma busca prévia, você poderia
fazer assim em modo de comandos:

\begin{verbatim}
     :%s,<Ctrl-r>/,novo-texto,g
\end{verbatim}

Observação: veja que estou trocando o delimitador da busca para deixar
claro o uso do registro de buscas ``/''

\section{Manipulando registros}
\label{Manipulando registros}

\begin{verbatim}
     :let @a=@_     : limpa o registro a
     :let @a=``''   : limpa o registro a
     :let @a=@"     : salva registro sem nome *N*
     :let @*=@a     : copia o registro para o buffer de colagem
     :let @*=@:     : copia o ultimo comando para o buffer de colagem
     :let @*=@/     : copia a última busca para o buffer de colagem
     :let @*=@%     : copia o nome do arquivo para o buffer de colagem
     :reg           : mostra o conteúdo de todos os registros
\end{verbatim}

Em modo de inserção

\begin{verbatim}
     <C-R>-         : Insere o registro de pequenas deleções
     <C-R>[0-9a-z]  : Insere registros 0-9 e a-z
     <C-R>%         : Insere o nome do arquivo
     <C-R>=somevar  : Insere o conteúdo de uma variável
\end{verbatim}


Um exemplo: pré-carregando o nome do arquivo no registro \verb+n+.

coloque em seu \verb+~/.vimrc+

\begin{verbatim}
     let @n=@%
\end{verbatim}

Como foi atribuído ao registro \verb+n+ o conteúdo de @\%, ou seja, o nome
do arquivo, você pode fazer algo assim em modo de inserção:

\begin{verbatim}
     Ctrl-r n
\end{verbatim}

E o nome do arquivo será inserido

\section{Listando os registros atuais}
\label{Listando os registros atuais}
Digitando o comando

\begin{verbatim}
     :reg
\end{verbatim}

ou ainda

\begin{verbatim}
     :ls
\end{verbatim}

O Vim mostrará os registros numerados e nomeados atualmente em uso

\section{Listando arquivos abertos}
\label{Listando arquivos abertos}
Suponha que você abriu vários arquivos txt assim:

\begin{verbatim}
     vim *.txt
\end{verbatim}

Para listar os arquivos aberto faça:

\begin{verbatim}
     :buffers
\end{verbatim}

Usando o comando acima o Vim exibirá a lista de todos os arquivos
abertos, após exibir a lista você pode escolher um dos arquivos da
lista, algo como:

\begin{verbatim}
     :buf 3
\end{verbatim}

Para editar arquivos em sequência faça as alterações no arquivo atual
e acesso o próximo assim:

\begin{verbatim}
     :wn
\end{verbatim}

O comando acima diz \verb|`grave' --> w|  e próximo \verb|`next' --> n|

\section{Dividindo a janela com o próximo arquivo da lista de buffers}
\label{Dividindo a janela com o próximo arquivo da lista de buffers}

\begin{verbatim}
     :sn
\end{verbatim}

O comando acima é uma abreviação de {\em split next}, ou seja, dividir e próximo.

\section{Como colocar um pedaço de texto em um registro?}
\label{Como colocar um pedaço de texto em um registro?}

\begin{verbatim}
     <Esc> ...... vai para o modo normal
     "a10j ...... coloca no registro `a' as próximas 10 linhas `10j'
\end{verbatim}

Para usar você pode:

\begin{verbatim}
     <Esc> ...... para ter certeza que está em modo normal
     "ap ........ registro a `paste', ou seja, cole
\end{verbatim}

Em modo de inserção faça:

\begin{verbatim}
     Ctrl-r a
\end{verbatim}

\section{Como criar um registro em modo visual?}
\label{Como criar um registro em modo visual?}
Inicie a seleção visual com o atalho

\begin{verbatim}
     Shift-v ..... seleciona linhas inteiras
\end{verbatim}

pressione a letra ``\verb|j|'' até chegar ao ponto desejado, agora faça

\begin{verbatim}
     "ay
\end{verbatim}

pressione ``\verb|v|'' para sair do modo visual

\section{Como definir um registro no vimrc?}
\label{Como definir um registro no vimrc?}

Se você não sabe ainda como editar preferências no Vim
leia antes o capítulo \ref{cha:Como editar preferências no Vim}. \\


Você pode criar uma variável no vimrc assim:

\begin{verbatim}
     let var="foo" ...... define foo para var
     echo var ........... mostra o valor de var
\end{verbatim}

Pode também dizer ao Vim algo como...

\begin{verbatim}
     :let @d=strftime("c")<Enter>
\end{verbatim}

Neste caso estou dizendo a ele que guarde na variável `d' at d,
o valor da data do sistema `strftime(``c'')' ou então cole isto no
vimrc:

\begin{verbatim}
     let @d=strftime("c")<cr>
\end{verbatim}

A diferença entre digitar diretamente um comando e adiciona-lo ao
vimrc é que uma vez no vimrc o registro em questão estará sempre
disponível, observe também as sutis diferenças, um {\tt Enter} inserido
manualmente é apenas uma indicação de uma ação que você fará
pressionando a tecla especificada, já o comando mapeado vira
``\verb|<cr>|'', veja ainda que no vimrc os dois pontos ``\verb|:|''
somem.

Pode mapear tudo isto

\begin{verbatim}
     let @d=strftime("c")<cr>
     imap ,d <cr-r>d
     nmap ,d "dp
\end{verbatim}

As atribuições acima correspondem a:

\begin{enumerate}
 \item  Guarda a data na variável ``d''
 \item  Mapeamento para o modo de inserção ``imap'' digite ,d
 \item  Mapeamento para o modo normal ``nmap'' digite ,d
\end{enumerate}

E digitar ,d normalmente

Desmistificando o strftime
\begin{verbatim}
     " d=dia m=mes Y=ano H=hora M=minuto c=data-completa
     :h strftime ........ ajuda completa sobre o comando
\end{verbatim}

e inserir em modo normal assim:

\begin{verbatim}
     "dp
\end{verbatim}

ou usar em modo de inserção assim

\begin{verbatim}
     Ctrl-r d
\end{verbatim}

\section{Como selecionar blocos verticais de texto?}
\label{Como selecionar blocos verticais de texto?}

\begin{verbatim}
     Ctrl-v
\end{verbatim}

agora use as letras h,l,k,j como setas de direção até finalizar
podendo guardar a seleção em um registro que vai de ``a'' a ``z'' exemplo:

\begin{verbatim}
     "ay
\end{verbatim}

Em modo normal você pode fazer assim para guardar um parágrafo inteiro em um registro

\begin{verbatim}
     "ayip
\end{verbatim}

O comando acima quer dizer

\begin{verbatim}
     para o registro ``a'' ......  "a
     copie ......................  ``y''
     o parágrafo atual .......... ``inner paragraph''
\end{verbatim}

\section{Referências}
\label{Referências}

\begin{itemize}
   \item \url{http://rayninfo.co.uk/vimtips.html}
   \item \url{http://aprendolatex.wordpress.com}
   \item \url{http://pt.wikibooks.org/wiki/Latex}
\end{itemize}

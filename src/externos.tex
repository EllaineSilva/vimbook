%%%%%%%%%%%%%%%%%%%%%%%%%%%%%%%%%%%%%%%%%%%%%%%%%%%%%%%%%%%%%%%%%%%%%%%%
% vim:enc=utf-8:ts=5:sw=5:et:ff=unix:
%%%%%%%%%%%%%%%%%%%%%%%%%%%%%%%%%%%%%%%%%%%%%%%%%%%%%%%%%%%%%%%%%%%%%%%%

\chapter{Comandos Externos}
O Vim permite executar comandos externos para processar ou filtrar o
conteúdo de um arquivo. De forma geral, fazemos isso digitando (no
modo normal):

\begin{verbatim}
     :!ls .... visualiza o conteúdo do diretório
\end{verbatim}

Lembrando que anexando um simples ponto, a saída do comando torna-se o 
documento que está sendo editado:

\begin{verbatim}
     :.!ls .... imprime na tela o conteúdo do diretório
\end{verbatim}

A seguir, veja alguns exemplos de utilização:

\section{Ordenando}
Podemos usar o comando {\em sort} que ordena o conteúdo de um arquivo dessa forma:

\begin{verbatim}
     :5,15!sort ..... ordena da linha 5 até a linha 15
\end{verbatim}

O comando acima ordena da linha 5 até a linha 15.

O comando {\em sort} existe tanto no Windows quanto nos sistemas Unix.
Digitando simplesmente {\em sort}, sem argumentos, o comportamento padrão
é de classificar na ordem alfabética (baseando-se na linha inteira).
Para mais informações sobre argumentos do comando {\em sort}, digite:

\begin{verbatim}
     sort --help ou man sort (no Unix) ou
     sort /? (no Windows).
\end{verbatim}

\section{Removendo linhas duplicadas}

\begin{verbatim}
     :%!uniq
\end{verbatim}

O caractere ``\%'' representa a região equivalente ao arquivo atual inteiro.
A versão do Vim 7 em diante tem um comando {\em sort} que permite remover
linhas duplicadas {\em uniq} e ordenar, sem a necessidade de usar comandos
externos, para mais detalhes:

\begin{verbatim}
     :h sort
\end{verbatim}

\section{Ordenando e removendo linhas duplicadas no Vim 7}

\begin{verbatim}
     :sort u
\end{verbatim}

Quando a ordenação envolver números faz-se:

\begin{verbatim}
     :sort n
\end{verbatim}

\section{{\em Beautifiers}}

A maior parte das linguagens de programação possui ferramentas
externas chamadas {\em beautifiers}, que servem para embelezar o código,
através da indentação e espaçamento. Por exemplo, para embelezar um
arquivo HTML é possível usar a ferramenta ``tidy\footnote{http://tidy.sourceforge.net/}'', do W3C:

\begin{verbatim}
     :%!tidy
\end{verbatim}

\section{Editando comandos longos no Linux}
\label{Editando comandos longos no Linux}

É comum no ambiente GNU/Linux a necessidade de digitar comandos longos
no terminal, para facilitar esta tarefa pode-se seguir estes passos:

\begin{enumerate}
     \item Definir o Vim como editor padrão do sistema editando 
           o arquivo `{\tt .bashrc}\footnote{Arquivo de configuração do bash}':
           \begin{verbatim}
               #configura o vim como editor padrão
               export EDITOR=vim
               export VISUAL=vim
            \end{verbatim}
      \item No terminal usar a combinação de teclas `{\tt Ctrl-x-e}'.
            Esta combinação de teclas abre o editor padrão do sistema
            onde se deve digitar o comando longo, ao sair do editor 
            o terminal executa o comando editado.
\end{enumerate}

\section{Compilando e verificando erros}
\vimhelp{cwindow quickfix-window}

Se o seu projeto já possui um {\tt Makefile}, então você pode fazer uso do comando
{\tt :make} para poder compilar seus programas no conforto de seu Vim:

\begin{verbatim}
     :make
\end{verbatim}

A vantagem de fazer isso é poder usar outra ferramenta bastante interessante, a janela
de {\em quickfix}:

\begin{verbatim}
     :cw[indow]
\end{verbatim}

O comando {\tt cwindow} abrirá uma janela em um {\em split} horizontal com a
listagem de erros e {\em warnings}.  Você poderá navegar pela lista usando os
cursores e ir diretamente para o arquivo e linha da ocorrência.

Modificando o compilador, o comando {\tt make} pode mudar sua ação.

\begin{verbatim}
	:compiler javac
	:compiler gcc
	:compiler php
\end{verbatim}

Note que {\em php} não tem um compilador. Logo, quando executado, o {\tt make} irá verificar
por erros de sintaxes.

\begin{verbatim}
	:compiler
\end{verbatim}

O comando acima lista todos os compiladores suportados.

\section{Grep}
\label{sec:Grep}
\vimhelp{grep quickfix-window cw}

Do mesmo jeito que você usa {\tt grep} na sua linha de comando você pode usar
o {\tt grep} interno do Vim. Exatamente do mesmo jeito:

\begin{verbatim}
     :grep <caminho> <padrão> <opções>
\end{verbatim}

Use a janela de {\em quickfix}\footnote{{\tt :cope}} aqui também para exibir os resultados do {\tt
grep} e poder ir diretamente a eles.

\section{Indent}

{\tt Indent}\footnote{http://www.gnu.org/software/indent} 
é um programa que indenta seu código fonte de acordo com os padrões configurados
no seu arquivo {\tt HOME/.indent.pro}. Vou pressupor que você já saiba usar o {\tt indent}
e como fazer as configurações necessárias para ele funcionar, então vamos ao funcionamento 
dele no Vim:

Para indentar um bloco de código, primeiro selecione-o com o modo {\em visual line} (com {\tt V}), 
depois é só entrar com o comando como se fosse qualquer outro comando externo:
\begin{verbatim}
     :!indent
\end{verbatim}

No caso, como foi selecionado um bloco de código, irão aparecer alguns caracteres extras, 
mas o procedimento continua o mesmo:
\begin{verbatim}
     :'<,'>!indent
\end{verbatim}


\section{Calculadora Científica com o Vim}
\label{sec:Calculadora Científica com o Vim}

Para usar a função de Calculadora Científica no Vim usamos uma ferramenta
externa, que pode ser o comando `{\tt bc}' do GNU/Linux, ou uma linguagem de
programação como {\em Python} ou {\em Ruby}, veremos como habilitar a
calculadora usando o {\em Python}. Obviamente esta linguagem de programação
deve estar instalada no sistema em que se deseja usar seus recursos.  Deve-se
testar se a versão do Vim tem suporte ao Python ``\verb+:version+'', em seguida
colocam-se os mapeamentos no ``.vimrc''.

\begin{verbatim}
     :command! -nargs=+ Calc :py print <args>
     :py from math import *
\end{verbatim}

Feito isto pode-se usar o comando ``{\tt :Calc}'' como visto abaixo:

\begin{verbatim}
     :Calc pi
     :Calc cos(30)
     :Calc pow(5,3)
     :Calc 10.0/3
     :Calc sum(xrange(1,101))
     :Calc [x**2 for x in range(10)] 
\end{verbatim}

\section{Editando saídas do Shell}
\label{sec:Editando saídas do Shell}

Muitas vezes, precisamos manipular saídas do shell antes de enviá-las por e-mail, reportar ao chefe ou até mesmo 
salvá-las. Utilizando

\begin{verbatim}
     vim -
     ou
     gvim -
\end{verbatim}

a saída do Shell é redirecionada para o (G)Vim automaticamente, não sendo necessário redirecioná-la para um arquivo
temporário e, logo após, abrí-lo para editá-lo e modificá-lo.

\section{Log do Subversion}

A variável de ambiente {\em \$SVN\_EDITOR} pode ser usada para se especificar o caminho para o editor de texto de
sua preferência, a fim de usá-lo na hora de dar um {\em commit} usando o {\em subversion}.

\begin{verbatim}
     export SVN_EDITOR=/usr/bin/vim
     svn commit
\end{verbatim}

Será aberto uma sessão no Vim, que depois de salva, será usada para LOG do commit.

\section{Referências}

\begin{itemize}
 \item \url{http://www.dicas-l.com.br/dicas-l/20070119.php}
 \item \url{http://vim.wikia.com/wiki/Scientific_calculator}
 \item \url{http://docs.python.org/library/cmath.html}
 \item \url{http://docs.python.org/library/math.html}
\end{itemize}

%%%%%%%%%%%%%%%%%%%%%%%%%%%%%%%%%%%%%%%%%%%%%%%%%%%%%%%%%%%%%%%%%%%%%%%%%%%%%%%%

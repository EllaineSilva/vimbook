%%%%%%%%%%%%%%%%%%%%%%%%%%%%%%%%%%%%%%%%%%%%%%%%%%%%%%%%%%%%%%%%%%%%%%%%
% vim:enc=utf-8:ts=5:sw=5:et:ff=unix:
%%%%%%%%%%%%%%%%%%%%%%%%%%%%%%%%%%%%%%%%%%%%%%%%%%%%%%%%%%%%%%%%%%%%%%%%

\chapter{Como Editar Preferências no Vim}\label{cha:Como editar preferências no Vim}

O arquivo de preferências do Vim é nomeado {\tt vimrc}, um arquivo oculto que
normalmente encontra-se no diretório de trabalho ({\em home}) do usuário:

\begin{verbatim}
     ~/.vimrc
     /home/usuario/.vimrc
\end{verbatim}

No sistema operacional Windows o arquivo costuma ser:

\begin{verbatim}
     ~\_vimrc
     c:\documents and settings\usuario\_vimrc
\end{verbatim}

\section{Onde colocar {\em plugins} e temas de cor}
\label{Onde colocar plugins e temas de cor}

No Windows deve haver uma pasta chamada `{\tt vimfiles}' (caso não exista
deve-se criá-la), que fica em:

\begin{verbatim}
     c:\documents and settings\usuario\vimfiles
\end{verbatim}

No GNU/Linux a pasta de arquivos do Vim é chamada {\tt .vim}, comumente
armazenada em

\begin{verbatim}
     /home/user/.vim
\end{verbatim}

Tanto em {\tt .vim} como {\tt vimfiles} encontram-se usualmente as seguintes
pastas:

\begin{verbatim}
     vimfiles ou .vim
        |
        +--color
        |
        +--doc
        |
        +--syntax
        |
        +--plugin
\end{verbatim}

Os {\em plugins}, como se pode deduzir, devem ser colocados no diretório
denominado {\tt plugin}. {\Large \ding{45}} Na seção Plugins~\ref{Plugins} (p.
\pageref{Plugins}) estão descritos alguns {\em plugins} para o Vim.

\section{Comentários }
\label{Comentários }

Comentários são linhas que são ignoradas pelo interpretador Vim e servem
normalmente para descrição de comandos e ações, deixando portanto mais legível
e didático o arquivo de configuração. Uma linha é um comentário se seu
primeiro caractere é uma aspa `\verb+"+':

\begin{verbatim}
     " linhas começadas com aspas são comentários
     " e portanto serão ignoradas pelo Vim
\end{verbatim}

Recomenda-se usar comentários ao adicionar ou modificar comandos no arquivo
{\tt vimrc}, pois assim torna-se mais fácil futuras leituras e modificações
neste arquivo.

\section{Efetivação das alterações no {\tt vimrc}}
\label{Efetivação das alterações no vimrc}

As alterações no {\tt vimrc} só serão efetivadas na próxima vez que o Vim for
aberto, a não ser que o recarregamento do arquivo de configuração seja
instruído explicitamente:

\begin{verbatim}
     :source ~/vimrc ....... se estiver no GNU/Linux
     :source ~/_vimrc ...... caso use o Windows
     :so arquivo ........... `so' é uma abreviação de `source'
\end{verbatim}

\section{{\em Set}}
\label{Set}

Os comandos {\tt set}, responsáveis por atribuir valores à variáveis,
 podem ser colocados no \verb|.vimrc|:

\begin{verbatim}
     set nu
\end{verbatim}

ou digitados como comandos:

\begin{verbatim}
     :set nu
\end{verbatim}


%%%%%%%%%%%%%%%%%%%%%%%%%%%%%%%%%%%%%%%%%%%%%%%%%%%%%%%%%%%%%%
% ESTE TRECHO NÃO PODE FICAR INDENTADO POIS SE ASSIM O FOR  %%
% ELE NÃO CABERÁ NA PÁGINA!!!!                              %%
%%%%%%%%%%%%%%%%%%%%%%%%%%%%%%%%%%%%%%%%%%%%%%%%%%%%%%%%%%%%%%
\index{cores!esquemas}
\begin{verbatim}
set number ............... "mostra numeração de linhas
set nu ................... "simplificação de `number'
set showmode ............. "mostra o modo em que estamos
set showcmd .............. "mostra no status os comandos inseridos
set tabstop=4 ............ "tamanho das tabulações
set ts=4 ................. "simplificação de `tabstop'
set shiftwidth=4 ......... "quantidade de espaços de uma tabulação
set sw=4 ................. "simplificação de `shiftwidth'
syntax on ................ "habilita cores
syn on ................... "simplificação de `syntax'
colorscheme tema ......... "esquema de cores `syntax highlight'
autochdir ................ "configura o diretório de trabalho
set hls .................. "destaca com cores os termos procurados
set incsearch ............ "habilita a busca incremental
set ai ................... "auto identação
set aw ................... "salva automaticamente ao trocar de `buffer'
set ignorecase ........... "ignora maiúsculas e minúsculas nas buscas
set ic ................... "simplificação de ignorecase
set smartcase ............ "numa busca em maiúsculo habilita `case'
set scs .................. "sinônimo de `smartcase'
set backup ............... "habilita a criação de arquivos de backup
set bk ................... "simplificação de `backup'
set backupext=.backup .... "especifica a extensão do arquivo de backup
set bex=.backup .......... "simplificação de backupext
set backupdir=~/.backup,./ "diretório(s) para arquivos de backup
set bdir ................. "simplificação de `backupdir'
set nobackup ............. "evita a criação de arquivos de backup
ste nobk ................. "simplificação de `nobackup'
set cursorline ........... "abreviação de cursor line (destaca linha atual)
set cul .................. "simplificação de `cursorline'
set ttyfast .............. "melhora o redraw de janelas.
set columns=88 ........... "deixa a janela com 88 colunas.
set mousemodel=popup ..... "exibe o conteúdo de folders e sugestões spell
set viminfo=%,'50,\"100,/100,:100,n "armazena opções (buffers)
\end{verbatim}

\section{Ajustando o parágrafos em modo normal} % (fold)
\label{sec:Ajustando o parágrafos em modo normal}
\index{parágrafo!ajustar}

O comando `\verb|gqap|' ajusta o parágrafo atual em modo normal. 
{\Large {\ding{45}}} usando a opção `{\tt :set nojoinspaces}' o vim 
colocará dois espaços após o ponto final ao se ajustar os parágrafos.
geralmente usamos `\verb+^I+' para representar uma tabulação        
<Tab>, e `\verb+$+' para indicar o fim de linha. Mas é possível     
customizar essas opções. sintaxe:                          

\begin{verbatim}
    
     set listchars=key:string,key:string                        

      - eol:{char} 
      Define o caracter a ser posto depois do fim da linha  

      - tab:{char1}{char2} 
      O tab é mostrado pelo primeiro caracter {char1} e     
      seguido por {char2}                                   

      - trail:{char}                                            
      Esse caracter representa os espaços em branco.        
                                                                
      - extends:{char}                                          
      Esse caracter representa o início do fim da linha      
      sem quebrá-la                                          
      Está opção funciona com a opção nowrap habilitada       
                                                                
     "exemplo 1:
     "set listchars=tab:>-,trail:.,eol:#,extends:@
     
     "exemplo 2:
     "set listchars=tab:>-
     
     "exemplo 3:
     "set listchars=tab:>-
     
     "exemplo 4:
     set nowrap    "Essa opção desabilita a quebra de linha
     "set listchars=extends:+
     
     Caso esteja usando o gvim pode setar um esquema de cores
     set colo desert
\end{verbatim}

% fim da seção Ajustando o parágrafos em modo normal

\section{Exibindo caracteres invisíveis}
\label{Exibindo caracteres invisíveis}

\begin{verbatim}
     :set list
\end{verbatim}

\section{Definindo macros previamente}
\label{Definindo macros previamente}
Definindo uma macro de nome \verb|s| para ordenar e retirar linhas duplicadas

\begin{verbatim}
     let @s=":sort u"
\end{verbatim}

Para executar a macro \verb|s| definida acima faça:

\begin{verbatim}
     @s
\end{verbatim}

O Vim colocará no comando

\begin{verbatim}
     :sort -u
\end{verbatim}

Bastando pressionar \verb|<Enter>|.
Observação: esta macro prévia pode ficar no {\tt vimrc} ou ser digitada em comando ``:''


\begin{verbatim}
     :5,20sort u
     "da linha 5 até a linha 20 ordene e retire duplicados
     
     :sort n
     " ordene meu documento considerando números
     " isto é útil pois se a primeira coluna contiver
     " números a ordenação pode ficar errada caso não usemos
     " o parâmetro ``n''
\end{verbatim}

\section{Mapeamentos}\label{Mapeamentos}
\vimhelp{key-mapping, mapping}

Mapeamentos permitem criar atalhos de teclas para quase tudo. Tudo depende da
criatividade do usuário e do quanto conhece o Vim, com eles podemos controlar ações
com quaisquer teclas, mas antes temos que saber que para criar mapeamentos,
precisamos conhecer a maneira de representar as teclas e combinações. Alguns
exemplos:

\begin{verbatim}
     tecla ....... tecla mapeada
     <c-x> ....... Ctrl-x
     <left> ...... seta para a esquerda
     <right> ..... seta para a direita
     <c-m-a> ..... Ctrl-Alt-a
     <cr> ........ Enter
     <Esc> ....... Escape
     <leader> .... normalmente \
     <bar> ....... | pipe
     <cword> ..... palavra sob o cursor
     <cfile> ..... arquivo sob o cursor
     <cfile> ..... arquivo sob o cursor sem extensão
     <sfile> ..... conteúdo do arquivo sob o cursor
     <left> ...... salta um caractere para esquerda
     <up> ........ equivale clicar em `seta acima'
     <m-f4> ...... a tecla alt -> m  mais a tecla f4
     <c-f> ....... Ctrl-f
     <bs> ........ backspace
     <space> ..... espaço
     <tab> ....... tab
\end{verbatim}

Para ler mais sobre atalhos de tecla no Vim acesse 

\begin{verbatim}
     :h index
\end{verbatim}

No Vim podemos mapear uma tecla para o modo normal, realizando
determinada operação e a mesma tecla pode desempenhar outra função
qualquer em modo de inserção ou comando, veja:

\begin{verbatim}
     " mostra o nome do arquivo com o caminho
     map <F2> :echo expand("%:p")
     " insere um texto qualquer
     imap <F2> Nome de uma pessoa
\end{verbatim}

A única diferença nos mapeamentos acima é que o mapeamento para modo de
inserção começa com `{\tt i}', assim como para o modo ``comando'' `{\tt :}'
começa com `{\tt c}' no caso `{\tt cmap}'.

\subsection{Recarregando o arquivo de configuração}
\label{sec:Recarregando o arquivo de configuração}

Cada alteração no arquivo de configuração do Vim só terá efeito na próxima vez que você
abrir o Vim a menos que você coloque isto dentro do mesmo

\begin{verbatim}
     " recarregar o vimrc
     " Source the .vimrc or _vimrc file, depending on system
     if &term == "win32" || "pcterm" || has("gui_win32")
        map ,v :e $HOME/_vimrc<CR>
        nmap <F12> :<C-u>source ~/_vimrc <BAR> echo "Vimrc recarregado!"<CR>
     else
        map ,v :e $HOME/.vimrc<CR>
        nmap <F12> :<C-u>source ~/.vimrc <BAR> echo "Vimrc recarregado!"<CR>
     endif
\end{verbatim}

Agora basta pressionar ``\verb|<F12>|'' em modo normal e as alterações passam a valer
instantaneamente, e para chamar o {\tt vimrc} basta usar.

\begin{verbatim}
     ,v
\end{verbatim}

Os mapeamentos abaixo são úteis para quem escreve códigos HTML, permitem
inserir caracteres reservados do HTML usando uma barra invertida para proteger
os mesmos, o Vim substituirá os ``barra alguma coisa'' pelo caractere
correspondente.

\begin{verbatim}
     inoremap \&amp; &amp;amp;
     inoremap \&lt; &amp;lt;
     inoremap \&gt; &amp;gt;
     inoremap \. &amp;middot;
\end{verbatim}

O termo {\em inoremap} significa: em modo de inserção não remapear, ou seja
ele mapeia o atalho e não permite que o mesmo seja remapeado, e o
mapeamento só funciona em modo de inserção, isso significa que um atalho
pode ser mapeado para diferentes modos de operação. \\

Veja este outro mapeamento:

\begin{verbatim}
     map <F11> <Esc>:set nu!<cr>
\end{verbatim}

Permite habilitar ou desabilitar números de linha do arquivo corrente.
A exclamação ao final torna o comando booleano, ou seja, se a
numeração estiver ativa será desabilitada, caso contrário será
ativada. O ``\verb|<cr>|'' ao final representa um {\tt Enter}.

\subsection{Limpando o ``registro'' de buscas}\label{Limpando o ``registro'' de buscas}

A cada busca, se a opção `{\tt hls}'\footnote{hls é uma abreviação de hightlight search}
estiver habilitada o Vim faz um destaque colorido, mas se você quiser limpar pode fazer este
mapeamento

\begin{verbatim}
     nno <S-F11> <Esc>:let @/=""<CR>
\end{verbatim}

É um mapeamento para o modo normal que faz com que a combinação de
teclas \verb|Shift-F11| limpe o ``registro'' de buscas

\subsection{Destacar palavra sob o cursor }
\label{Destacar palavra sob o cursor }

\begin{verbatim}
     nmap <s-f> :let @/=">"<CR>
\end{verbatim}

O atalho acima \verb|s-f| corresponde a \verb|Shift-f|.

\subsection{Remover linhas em branco duplicadas }
\label{Remover linhas em branco duplicadas }

\begin{verbatim}
     map ,d <Esc>:%s/\(^\n\{2,}\)/\r/g<cr>
\end{verbatim}

No mapeamento acima estamos associando o atalho:

\begin{verbatim}
     ,d
\end{verbatim}

\dots~à ação desejada, fazer com que linhas em branco sucessivas sejam
substituídas por uma só linha em branco, vejamos como funciona:

\begin{verbatim}
     map ......... mapear
     ,d .......... atalho que quermos
     <Esc> ....... se estive em modo de inserção sai
     : ........... em modo de comando
     % ........... em todo o arquivo
     s ........... substitua
     \n .......... quebra de linha
     {2,} ........ duas ou mais vezes
     \r .......... trocado por \r Enter
     g ........... globalmente
     <cr> ........ confirmação do comando
\end{verbatim}

As barras invertidas podem não ser usadas se o seu Vim estiver com a opção
{\em magic} habilitada

\begin{verbatim}
     :set magic
\end{verbatim}

Por acaso este é um padrão portanto tente usar assim pra ver se funciona

\begin{verbatim}
     map ,d :%s/\n{2,}/\r/g<cr>
\end{verbatim}

\subsection{Mapeamento para Calcular Expressões}
\label{sub:Mapeamento para Calcular Expressões}

Os mapeamentos abaixo exibem o resultado das quatro operações básicas (soma,
subtração, multiplicação e divisão). O primeiro para o modo normal no qual
posiciona-se o cursor no primeiro caractere da expressão tipo ``{\tt 5*9}'' e em
seguida pressiona-se ``{\tt Shift-F1}'', o segundo para o modo {\em insert} em que, após
digitada a expressão pressiona-se o mesmo atalho.

\begin{verbatim}
    " calculadora
    map <s-f1> <esc>0"myEA=<c-r>=<c-r>m<enter><esc>
    imap <s-f1> <space><esc>"myBEa=<c-r>=<c-r>m<enter><del>
\end{verbatim}

Para efetuar cálculos com maior precisão e também resolver problemas como
potências raízes, logaritmos pode-se mapear comandos externos, como a
biblioteca matemática da linguagem de programação Python.
\href{http://vim.wikia.com/wiki/Scientific\_calculator}{Neste link}~\cite{CientificCalculator} há um
manual que ensina a realizar este procedimento, ou acesse o capítulo \ref{Uma
calculadora diferente} na página~ \pageref{Uma calculadora diferente}.


\subsection{Mapeamentos globais}


Podemos fazer mapeamentos globais ou que funcionam em apenas um modo:

\begin{verbatim}
     map  - funciona em qualquer modo
     nmap - apenas no modo Normal
     imap - apenas no modo de Inserção
\end{verbatim}

Mover linhas com {\tt Ctrl-$\downarrow$} ou {\tt Ctrl-$\uparrow$}:

\begin{verbatim}
     " tem que estar em modo normal!
     nmap <C-Down> ddp
     nmap <C-Up> ddkP
\end{verbatim}

Salvando com uma tecla de função:

\begin{verbatim}
     " salva com F9
     nmap <F9> :w<cr>
     " F10 - sai do Vim
     nmap <F10> <Esc>:q<cr>
\end{verbatim}

\subsection{Convertendo as iniciais de um documento para maiúsculas}
\label{Convertendo as iniciais de um documento para maiúsculas}

\begin{verbatim}
     " MinusculasMaiusculas: converte a primeira letra de cada
     " frase para MAIÚSCULAS
     nmap ,mm :%s/\C\([.!?][])"']*\($\|[ ]\)\_s*\)\(\l\)/\1\U\3/g<CR>
     " Caso queira confirmação coloque uma letra ``c'' no final da 
     " linha acima:
     " (...) \3/gc<CR>
\end{verbatim}

\section{Autocomandos }
\label{Autocomandos }
\vimhelp{autocmd.txt}

Autocomandos habilitam comandos automáticos para situações
específicas. Para executar determinada ação ao
iniciar um novo arquivo o autocomando deverá obedecer este padrão:

\begin{verbatim}
     au BufNewFile tipo ação
\end{verbatim}

Veja um exemplo:

\begin{verbatim}
     au BufNewFile,BufRead *.txt source ~/.vim/syntax/txt.vim
\end{verbatim}

No exemplo acima o Vim aplica autocomandos para arquivos novos
``{\tt BufNewfile}'' ou existentes ``{\tt BufRead}'' terminados em \verb|txt|, e para
estes tipos carrega um arquivo de {\em syntax}\index{syntax}, ou seja, um esquema de cores
específico.

\begin{verbatim}
     " http://aurelio.net/doc/vim/txt.vim    coloque em ~/.vim/syntax
     au BufNewFile,BufRead *.txt source ~/.vim/syntax/txt.vim
\end{verbatim}

Para arquivos do tipo texto `{\tt *.txt}' use um arquivo de {\em syntax} em
particular.

O autocomando abaixo coloca um cabeçalho para {\em scripts} {\tt bash} caso a
linha 1 esteja vazia, observe que os arquivos em questão tem que ter a
extensão `{\tt .s}'.

\begin{verbatim}
     au BufNewFile,BufRead *.sh if getline(1) == "" | normal ,sh
\end{verbatim}

\subsection{Exemplo prático de autocomandos}
\label{sub:Exemplo prático de autocomandos}

Há situações em que é necessária a uniformização de ações, por exemplo, em
códigos Python deve-se manter um padrão para a indentação, ou será com espaços
ou será com tabulações, não se pode misturar os dois pois o interpretador retornaria um erro.
Outra situação em que misturar espaços com tabulações ocasiona erros é em
códigos \LaTeX, ao compilar o documento a formatação não sai como desejado.
Até que se perceba o erro leva um tempo.  Para configurar o vim de forma que
ele detecte este tipo de erro ao entrar no arquivo:

\begin{verbatim}
     au! VimEnter * match ErrorMsg /^\t\+/

     " explicação para o autocomando acima
     au! ............... automaticamente
     VimEnter .......... ao entrar no vim
     * ................. para qualquer tipo de arquivo
     match ErrorMsg .... destaque como erro
     / ................. inicio de um padrão
     ^ ................. começo de linha
     \t ................ tabulação
     \+ ................ uma vez ou mais
     / ................. fim do padrão de buscas
\end{verbatim}

Para evitar que este erro se repita, ou seja, que sejam 
adicionados no começo de linha tabulações no lugar de espaços
adiciona-se ao \textasciitilde/.vimrc

\begin{verbatim}
     set expandtab
\end{verbatim}

É perfeitamente possível um autocomando que faça direto a substituição de
tabulações por espaços, mas neste caso não é recomendado que o autocomando se
aplique a todos os tipos de aquivos.

\section{Funções}
\label{sec:Funções}

\subsection{Fechamento automático de parênteses}
\label{sec:Fechamento automático de parênteses}

\begin{verbatim}
     " --------------------------------------
     " Ativa fechamento automático para parêntese
     " Set automatic expansion of parenthesis/brackets
     inoremap ( ()<Esc>:call BC_AddChar(``)'')<cr>i
     inoremap { {}<Esc>:call BC_AddChar(``}'')<cr>i
     inoremap [ []<Esc>:call BC_AddChar(``]'')<cr>i
     `` '' ``''<Esc>:call BC_AddChar(``''")<cr>i
     "
     " mapeia Ctrl-j para pular fora de parênteses colchetes etc...
     inoremap <C-j> <Esc>:call search(BC_GetChar(), ``W'')<cr>a
     " Function for the above
     function! BC_AddChar(schar)
        if exists(``k'')
            let b:robstack = b:robstack . a:schar
        else
            let b:robstack = a:schar
        endif
     endfunction
     function! BC_GetChar()
        let l:char = b:robstack[strlen(b:robstack)-1]
        let b:robstack = strpart(b:robstack, 0, strlen(b:robstack)-1)
        return l:char
     endfunction
    
    '''Outra opção para fechamento de parênteses'''
    
     " Fechamento automático de parênteses
     imap { {}<left>
     imap ( ()<left>
     imap [ []<left>
    
     " pular fora dos parênteses, colchetes e chaves, mover o cursor
     " no modo de inserção
     imap <c-l> <Esc><right>a
     imap <c-h> <Esc><left>a
\end{verbatim}

\subsection{Função para barra de status}\label{Função para barra de status}

\begin{verbatim}
     set statusline=%F%m%r%h%w\
        [FORMAT=%{&ff}]\
        [TYPE=%Y]\
        [ASCII=\%03.3b]\
        [HEX=\%02.2B]\
        [POS=%04l,%04v][%p%%]\ [LEN=%L]
\end{verbatim}

Caso este código não funcione acesse
\href{http://vim.wikia.com/wiki/Writing\_a\_valid\_statusline}{este
link}~\cite{StatusLine}.


\subsection{Rolar outra janela}\label{Rolar outra janela}

Se você dividir janelas tipo

\begin{verbatim}
     Ctrl-w n
\end{verbatim}

pode colocar esta função no seu \verb|.vimrc|

\begin{verbatim}
     " rola janela alternativa
     fun! ScrollOtherWindow(dir)
     if a:dir == ``n''
        let move = ``>''
     elseif a:dir == ``p''
        let move = ``>''
     endif
     exec ``p'' . move . ``p''
     endfun
     nmap <silent> <M-Down> :call ScrollOtherWindow(``n'')<CR>
     nmap <silent> <M-Up> :call ScrollOtherWindow(``p'')<CR>
\end{verbatim}

Esta função é acionada com o atalho {\tt Alt-$\uparrow$} e {\tt Alt-$\downarrow$}.

\subsection{Função para numerar linhas}\label{Função para numerar linhas}
\index{linhas!numerar}

No site wikia há um código de função para
\href{http://vim.wikia.com/wiki/Number\_a\_group\_of\_lines}{numerar
linhas}~\cite{NumerarLinhas}

\subsection{Função para trocar o esquema de cores}
\index{cores!esquemas}

\begin{verbatim}
     function! <SID>SwitchColorSchemes()
       if exists(``e'')
        if g:colors_name == 'native'
          colorscheme billw
        elseif g:colors_name == 'billw'
          colorscheme desert
        elseif g:colors_name == 'desert'
          colorscheme navajo-night
        elseif g:colors_name == 'navajo-night'
          colorscheme  zenburn
        elseif g:colors_name == 'zenburn'
          colorscheme bmichaelsen
        elseif g:colors_name == 'bmichaelsen'
          colorscheme wintersday
        elseif g:colors_name == 'wintersday'
          colorscheme summerfruit
        elseif g:colors_name == 'summerfruit'
          colorscheme native
        endif
       endif
     endfunction
     map <silent> <F6> :call <SID>SwitchColorSchemes()<CR>
\end{verbatim}

baixe os esquemas
\href{http://nanasi.jp/old/colorscheme\_0.html}{neste link}~\cite{EsquemasDeCor}.

\subsection{Uma função para inserir cabeçalho de script}
\label{Uma função para inserir cabeçalho de script bash}
para chamar a função
basta pressionar, sh em modo normal

\begin{verbatim}
     " Cria um cabeçalho para scripts bash
     fun! InsertHeadBash()
        normal(1G)
        :set ft=bash
        :set ts=4
        call append(0, ``h'')
        call append(1, ``:'' . strftime("%a %d/%b/%Y hs %H:%M"))
        call append(2, "# ultima modificação:``(''%a %d/%b/%Y hs %H:%M"))
        call append(3, "# NOME DA SUA EMPRESA")
        call append(3, "# Propósito do script")
        normal($)
     endfun
     map ,sh :call InsertHeadBash()<cr>
\end{verbatim}

\subsection{Função para inserir cabeçalhos Python}
\label{Função para inserir cabeçalhos Python}

\begin{verbatim}
     " função para inserir cabeçalhos Python
     fun! BufNewFile_PY()
      normal(1G)
      :set ft=python
      :set ts=2
      call append(0, "#!/usr/bin/env python")
      call append(1, "# # -*- coding: ISO-8859-1 -*-")
      call append(2, ``:'' . strftime("%a %d/%b/%Y hs %H:%M"))
      call append(3, `` '' . strftime("%a %d/%b/%Y hs %H:%M"))
      call append(4, "# Instituicao: <+nome+>")
      call append(5, "# Proposito do script: <+descreva+>")
      call append(6, "# Autor: <+seuNome+>")
      call append(7, "# site: <+seuSite+>")
      normal gg
     endfun
     autocmd BufNewFile *.py call BufNewFile_PY()
     map ,py :call BufNewFile_PY()<cr>A
   
     " Ao editar um arquivo será aberto no último ponto em
     " que foi editado
   
     autocmd BufReadPost *
       \ if line('``\''``('''\``'') <= line(``$'') |
       \   exe ''normal g`\``" |
       \ endif
\end{verbatim}

\begin{verbatim}
     " Permite recarregar o Vim para que modificações no
     " Próprio vimrc seja ativadas com o mesmo sendo editado
     nmap <F12> :<C-u>source $HOME/.vimrc <BAR> echo "Vimrc recarregado!"<CR>
\end{verbatim}

Redimensionar janelas

\begin{verbatim}
     " Redimensionar a janela com
     " Alt-seta à direita e esquerda
     map <M-right> <Esc>:resize +2 <CR>
     map <M-left> <Esc>:resize -2 <CR>
\end{verbatim}

\subsection{Função para pular para uma linha}
\label{Função para pular para uma linha}

\begin{verbatim}
     "ir para linha
     " ir para uma linha específica
     function! GoToLine()
     let ln = inputdialog("ir para a linha...")
     exe ``:'' . ln
     endfunction
     "no meu caso o mapeamento é com Ctrl-l
     "use o que melhor lhe convier
     imap <S-l> <C-o>:call GoToLine()<CR>
     nmap <S-l> :call GoToLine()<CR>
\end{verbatim}

\subsection{Função para gerar backup}
\label{Função para gerar backup}

A função abaixo é útil para ser usada quando você vai editar um arquivo
gerando modificações significativas, assim você poderá restaurar o backup se necessário

\begin{verbatim}
     " A mapping to make a backup of the current file.
     fun! WriteBackup()
        let fname = expand("%:p") . "__" . strftime("%d-%m-%Y--%H.%M.%S")
        silent exe ":w " . fname
        echo "Wrote " . fname
     endfun
     nnoremap <Leader>ba :call WriteBackup()<CR>
\end{verbatim}

{\Large \ding{45}} O atalho ``{\tt <leader>}'' em geral é a barra 
invertida ``$\backslash$'', na dúvida ``{\tt :help <leader>}''.

\section{Como adicionar o Python ao {\em path} do Vim?}
\label{Como adicionar o Python ao path do Vim?}

Coloque o seguinte
\href{http://vim.wikia.com/wiki/Automatically\_add\_Python\_paths\_to\_Vim\_path}{script}~\cite{PythonPath}
 em:

\begin{verbatim}
     * ~/.vim/after/ftplugin/python.vim    (on Unix systems)
     %* $HOME/vimfiles/after/ftplugin/python.vim    (on Windows systems)
\end{verbatim}

\begin{verbatim}
     python << EOF
     import os
     import sys
     import vim
     for p in sys.path:
         # Add each directory in sys.path, if it exists.
         if os.path.isdir(p):
             # Command `set' needs backslash before each space.
             vim.command(r``s'' % (p.replace(`` '', r`` '')))
     EOF
\end{verbatim}

Isto lhe permite usar `{\tt gf}' ou {\tt Ctrl-w Ctrl-F} para abrir um arquivo sob o cursor

\section{Criando um menu}
\label{Criando um menu}

Como no Vim podemos ter infinitos comandos fica complicado memorizar tudo
é aí que entram os menus, podemos colocar nossos plugins e atalhos favoritos
em um menu veja este exemplo

\begin{verbatim}
     amenu Ferramentas.ExibirNomeDoTema :echo g:colors_name<cr>
\end{verbatim}

O comando acima diz:

\begin{verbatim}
     amenu ........................ cria um menu
     Ferramentas.ExibirNomeDoTema . Menu plugin submenu ExibirNomeDoTema
     :echo g:colors_name<cr> ...... exibe o nome do tema atual
\end{verbatim}

Caso haja espaços no nome a definir você pode fazer assim

\begin{verbatim}
     amenu Ferramentas.Exibir\ nome\ do\ tema :echo g:colors_name<cr>
\end{verbatim}

\section{Criando menus para um modo específico}
\label{Criando menus para um modo específico}

\begin{verbatim}
     :menu .... Normal, Visual e Operator-pending
     :nmenu ... Modo Normal
     :vmenu ... Modo Visual
     :omenu ... Operator-pending modo
     :menu! ... Insert e Comando
     :imenu ... Modo de inserção
     :cmenu ... Modo de comando
     :amenu ... Todos os modos
\end{verbatim}

\section{Exemplo de menu}
\label{Exemplo de menu}

\begin{verbatim}
     " cores
     menu T&emas.cores.quagmire :colo quagmire<CR>
     menu T&emas.cores.inkpot :colo inkpot<CR>
     menu T&emas.cores.google :colo google<CR>
     menu T&emas.cores.ir_black :colo ir_black<CR>
     menu T&emas.cores.molokai :colo molokai<CR>
     " Fontes
     menu T&emas.fonte.Inconsolata :set gfn=Inconsolata:h10<CR>
     menu T&emas.fonte.Anonymous :set anti gfn=Anonymous:h8<CR>
     menu T&emas.fonte.Envy\ Code :set anti gfn=Envy_Code_R:h10<CR>
     menu T&emas.fonte.Monaco :set gfn=monaco:h9<CR>
     menu T&emas.fonte.Crisp :set anti gfn=Crisp:h12<CR>
     menu T&emas.fonte.Liberation\ Mono :set gfn=Liberation\ Mono:h10<CR>
\end{verbatim}

{\Large \ding{45}} O comando ``{\tt :update}'' Atualiza o menu recém modificado.  Quando o comando
``{\tt :amenu}'' É usado sem nenhum argumento o Vim mostra os menus definidos
atualmente.  Para listar todas as opções de menu para `Plugin' por exemplo
digita-se no modo de comandos ``{\tt :amenu Plugin}''.

\section{Outros mapeamentos}
\label{Outros mapeamentos}

Destaca espaços e tabulações redundantes:

\begin{verbatim}
     highlight RedundantWhitespace ctermbg=red guibg=red
     match RedundantWhitespace /\s\+$\| \+\ze\t/
\end{verbatim}

Explicando com detalhes

\begin{verbatim}
     \s ..... espaço
     \+ ..... uma ou mais vezes
     $ ...... no final da linha
     \| ..... ou
     `` '' .. espaço (veja imagem acima)
     \+ ..... uma ou mais vezes
     \ze .... até o fim
     \t ..... tabulação
\end{verbatim}

Portanto a expressão regular acima localizará espaços ou tabulações no final de linha
e destacará em vermelho.


\begin{verbatim}
     "Remove espaços redundantes no fim das linhas
     map <F7> <Esc>mz:%s/\s\+$//g<cr>`z
\end{verbatim}

Um detalhe importante

\begin{verbatim}
     mz ... marca a posição atual do cursor para retornar no final do comando
     `z ... retorna à marca criada
\end{verbatim}

Se não fosse feito isto o cursor iria ficar na linha da última substituição!

\begin{verbatim}
     "Abre o vim explorer
     map <F6> <Esc>:vne .<cr><bar>:vertical resize -30<cr><bar>:set nonu<cr>
\end{verbatim}

Podemos usar ``Expressões Regulares\footnote{\url{http://guia-er.sourceforge.net}}'' em
buscas do Vim veja um exemplo para retirar todas as tags HTML

\begin{verbatim}
     "mapeamento para retirar tags HTML com Ctrl-Alt-t
     nmap <C-M-t> :%s/<[^>]*>//g <cr>
     " Quebra a linha atual no local do cursor com F2
     nmap <F2> a<CR><Esc>
     " join lines  -- Junta as linhas com Shift-F2
     nmap <S-F2> A<Del><Space>
\end{verbatim}

Para mais detalhes sobre buscas acesse o capítulo``\ref{cha:Buscas e substituições}
na página \pageref{cha:Buscas e substituições}''.

\section{Complementação com ``tab''}\label{Complementação com ``tab''}

\begin{verbatim}
     "Word completion
     "Complementação de palavras
     
     set dictionary+=/usr/dict/words
     set complete=.,w,k
     
     "------ complementação de palavras ----
     "usa o tab em modo de inserção para completar palavras
     
     function! InsertTabWrapper(direction)
        let col = col(``.'') - 1
        if !col || getline(``.'')[col - 1] !~ '\k'
           return ``>''
        elseif ``d'' == a:direction
           return ``>''
        else
           return ``>''
        endif
     endfunction
     
     inoremap <tab> <c-r>=InsertTabWrapper (``d'')<cr>
     inoremap <s-tab> <c-r>=InsertTabWrapper (``d'')<cr>
\end{verbatim}

\section{Abreviações}\label{Abreviações}

Abreviações habilitam auto-texto para o Vim. O seu funcionamento consiste de
três campos, o primeiro é o modo no qual a abreviação funcionará, o segundo é a
palavra que irá disparar a abreviação e o terceiro campo é a abreviação
propriamente dita. Para que em {\em modo de comando `:'} a palavra `salvar'
funcione para salvar os arquivos, adiciona-se a seguinte abreviação ao `{\tt
\textasciitilde/.vimrc}'. 

\begin{verbatim}
    cab salvar w
\end{verbatim}

Abaixo abreviações para o modo de inserção:

\begin{verbatim}
     iab slas Sérgio Luiz Araújo Silva
     iab Linux GNU/Linux
     iab linux GNU/Linux
\end{verbatim}

\section{Evitando arquivos de backup no disco}
\label{Evitando arquivos de backup no disco}

Nota-se em algumas situações que existem alguns arquivos com o mesmo nome dos
arquivos que foram editados, porém com um til (\textasciitilde) no final. Esses
arquivos são {\em backups} que o Vim gera antes de sobrescrever os arquivos, e
podem desde ocupar espaço significativo no disco rígido até representar falha
de segurança, como por exemplo arquivos {\tt .php\textasciitilde} que não são
interpretados pelo servidor web e expõem o código-fonte.

Para que os {\em backups} sejam feitos enquanto os arquivos estejam sendo escritos, porém não 
mantidos após terminar a escrita, utiliza-se no \verb|.vimrc|:

\begin{verbatim}
     set nobackup
     set writebackup
\end{verbatim}

Fonte: \url{http://eustaquiorangel.com/posts/520}

\section{Mantendo apenas um Gvim aberto}
\label{Mantenddo apenas um Gvim aberto}

Essa dica destina-se apenas à versão do Vim que roda no ambiente gráfico, ou
seja, o Gvim, pois ela faz uso de alguns recursos que só funcionam nesse
ambiente. A meta é criar um comando que vai abrir os arquivos indicados em abas
novas sempre na janela já existente. 

Para isso deve-se definir um {\em script} que esteja no seu
{\em path}\footnote{Diretórios nos quais o sistema busca pelos comandos} do sistema
(e que possa ser executado de alguma forma por programas do tipo {\em launcher}
no modo gráfico) que vai ser utilizado sempre que quisermos abrir nossos
arquivos dessa maneira. Para efeito do exemplo, o nome do arquivo será {\tt
tvim} (de {\em tabbed vim}), porém pode ser nomeado com o nome que for
conveniente.

A única necessidade para essa dica funcionar é a versão do Vim ter suporte para
o argumento {\tt --serverlist}, o que deve ser garantido nas versões presentes
na época em que esse documento foi escrito. Para fazer uma simples verificação
se o comando está disponível, deve ser digitado em um terminal:

\begin{verbatim}
     vim --serverlist
     gvim --serverlist
\end{verbatim}

Se ambos os comandos acima resultaram em erro, o procedimento não poderá ser
implementado. Do contrário, deve-se utilizar o comando que teve um retorno
válido ({\tt vim} ou {\tt gvim}) para a criar o {\em script}. Supondo que foi o
comando {\tt gvim} que não retornou um erro, criamos o {\em script} da seguinte
forma:

\begin{verbatim}
     #!/bin/bash
     if [ $# -ne 1 ]
     then
        echo "Sem arquivos para editar."
        exit
     fi
     gvim --servername $(gvim --serverlist | head -1) --remote-tab $1
\end{verbatim}

Desse modo, se for digitado {\tt tvim} sem qualquer argumento, é exibida a
mensagem de erro, do contrário, o arquivo é aberto na cópia corrente do Gvim,
em uma nova aba, por exemplo:

\begin{verbatim}
     tvim .vimrc
\end{verbatim}

Fonte: \url{http://eustaquiorangel.com/posts/477}

\section{Referências}
\label{Referências}
* \url{http://www.dicas-l.com.br/dicas-l/20050118.php}

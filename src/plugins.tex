%%%%%%%%%%%%%%%%%%%%%%%%%%%%%%%%%%%%%%%%%%%%%%%%%%%%%%%%%%%%%%%%%%%%%%%%
% vim:enc=utf-8:ts=5:sw=5:et:ff=unix:
%%%%%%%%%%%%%%%%%%%%%%%%%%%%%%%%%%%%%%%%%%%%%%%%%%%%%%%%%%%%%%%%%%%%%%%%

\chapter{Plugins}\label{Plugins}
``{\em Plugins}''\footnote{Plugins são recursos que se adicionam aos programas}
são um meio de estender as funcionalidades do Vim, há ``{\em plugins}'' para
diversas tarefas, desde wikis para o Vim até ferramentas de auxílio a navegação
em arquivos com é o caso do ``{\em plugin}''
\href{http://www.vim.org/scripts/script.php?script\_id=1658}{NerdTree}~\cite{PluginNerdTree}, que
divide uma janela que permite navegar pelos diretórios do sistema a fim de
abrir arquivos a serem editados.

\section{Como testar um plugin sem instalá-lo?}
\label{Como testar um plugin sem instala-lo?}

\begin{verbatim}
     :source <path>/<plugin>
\end{verbatim}

Caso o plugin atenda as necessidades, pode-se instalá-lo. Este
procedimento também funciona para temas de cor!


No GNU/Linux
\begin{verbatim}
     ~/.vim/plugin/
\end{verbatim}

No Windows

\begin{verbatim}
     ~/vimfiles/plugin/
\end{verbatim}

Obs: Caso ainda não exista o diretório, ele pode ser criado pelo próprio usuário

Exemplo no GNU/Linux

\begin{verbatim}
     + /home/usuario/
                 |
                 |
                 + .vim
                   |
                   |
                   + plugin
\end{verbatim}

Obs: Alguns plugins dependem da versão do Vim, para saber qual
a que está atualmente instalada:

\begin{verbatim}
     :ve[rsion]
\end{verbatim}

\section{Atualizando a documentação dos plugins}\label{sec:Atualizando a documentação dos plugins}
\vimhelp{helptags}
Caso seja adicionado algum arquivo de documentação em `\verb+~/.vim/doc+' pode-se 
gerar novamente as tags ``links'' para navegação de ajuda.

\begin{verbatim}
    :helptags $VIMRUNTIME/doc
    :helptags ~/.vim/doc
\end{verbatim}


\section{Plugin para \LaTeX}
\label{Plugin para LaTeX}

Um plugin completo para \LaTeX está acessível
\href{http://vim-latex.sourceforge.net/}{aqui}~\cite{PluginLatex}. Uma vez adicionado o plugin você pode
inserir seus {\em templates} em:

\begin{verbatim}
     ~/.vim/ftplugin/latex-suite/templates
\end{verbatim}


\section{Criando {\em folders} para arquivos \LaTeX}
\label{Criando folders para arquivos LaTeX}

\begin{verbatim}
     set foldmarker=\\begin,\\end
     set foldmethod=marker
\end{verbatim}

Adicionar marcadores ({\em labels}) às seções de um documento \LaTeX
\begin{verbatim}
     :.s/^\(\\section\)\({.*}\)/\1\2\r\\label\2
     
     : ........... comando
     / ........... inicia padrão de busca
     ^ ........... começo de linha
     \(palavra\) . agrupa um trecho
     \(\\section\) agrupa `\section'
     \\ .......... torna \ literal
     { ........... chave literal
     .* .......... qualquer caractere em qualquer quantidade
     } ........... chave literal
     / ........... finaliza parão de busca
     \1 .......... repeter o grupo 1 \(\\section\) 
     \2 .......... repete o grupo 2 \({.*\}\)
     \r .......... insere quebra de linha
     \\ .......... insere uma barra invertida
     \2 .......... repete o nome da seção
\end{verbatim}

\section{Criando seções \LaTeX}\label{Criando seções latex}
o comando abaixo substitui

\begin{verbatim}
     ==seção==
\end{verbatim}

   por

\begin{verbatim}
     \section{seção}
\end{verbatim}

\begin{verbatim}
     :.s/^==\s\?\([^=]*\)\s\?==/\\section{\1}/g
     
     : ......... comando
     . ......... linha atual
     s ......... substitua
     ^ ......... começo de linha
     == ........ dois sinais de igual
     \s\? ...... seguido ou não de espaço
     [^=] ...... não pode haver = (^ dentro de [] é negação)
     * ......... diz que o que vem antes pode vir zero ou mais vezes
     \s\? ...... seguido ou não de espaço
     \\ ........ insere uma barra invertida
     \1 ........ repete o primeiro trecho entre ()
\end{verbatim}

\section{Plugin para manipular arquivos}
Acesse o plugin neste
\href{http://www.vim.org/scripts/script.php?script\_id=2337#0.1.9}{link}.
Para entender este plugin acesse um vídeo
 \href{http://www.screencast.com/t/P6nJkJ0DE}{neste link}.


\section{Complementação de códigos}
\label{Complementação de códigos}
\index{snippets}

O ``{\em plugin}'' snippetsEmu é um misto entre complementação de códigos e
os chamados modelos ou {\em templates}. Insere um trecho de código pronto,
mas vai além disso, permitindo saltar para trechos do modelo inserido
através de um atalho configurável de modo a agilizar o trabalho do
programador. \href{http://www.vim.org/scripts/script.php?script\_id=1318}{Link para baixar}.

\subsection{Instalação}
\label{Instalação}

Um artigo ensinando como instalar o ``{\em plugin}'' snippetsEmu pode ser lido 
\href{http://vivaotux.blogspot.com/2008/03/instalando-o-plugin-snippetsemu-no-vim.html}{nesse artigo na internet}~\cite{SilvasnipptsEmu}.
Outro plugin muito interessante para complementação é o ``autocompletepopup'' que complementa 
mostrando um popup durante a digitação, o mesmo pode ser obtido 
\href{http://www.vim.org/scripts/script.php?script\_id=1879}{neste link}~\cite{NishidaAutoCompletePopup}, em seguida coloca-se esta 
linha ao vimrc:

\begin{verbatim}
     let g:AutoComplPop_CompleteoptPreview = 1
\end{verbatim}

A linha acima faz com que o vim abra uma janela pequena com a documentação
de cada método que está sendo digitado.

\section{Um wiki para o Vim}\label{sec:Um wiki para o Vim}
O ``{\em plugin}'' wikipot implementa um wiki para o Vim no qual você define
um ``link'' com a notação WikiWord, onde um ``link'' é uma palavra que
começa com uma letra maiúscula e tem outra letra maiúscula no meio
Obtendo o plugin  \href{http://www.vim.org/scripts/script.php?script\_id=1018}{neste link}.

\section{Acessando documentação do Python no Vim}\label{Acessando documentação do Python no Vim}
 Obtenha um plugin para esta tarefa em seu 
 \href{http://www.vim.org/scripts/script.php?script\_id=910}{site oficial}.

\section{Formatando textos planos com syntax}\label{Formatando textos planos com syntax}
Um plugin que adiciona syntaxe colorida a textos planos pode ser obtido
\href{http://www.vim.org/scripts/script.php?script\_id=2208&rating=helpful#1.3}{neste link}.
Veja como instalar o este plugin no capítulo \ref{sec:Um wiki para o Vim}.

\section{Movimentando em {\em camel case}}\label{Movimentando em camel case}
O {\em plugin} \href{http://www.vim.org/scripts/script.php?script\_id=1905}{{\tt
CamelCaseMotion}} auxilia a navegação em palavras em
\href{http://en.wikipedia.org/wiki/Camel\_case}{{\em camel case}} ou separadas
por sublinhados, através de mapeamentos similares aos que fazem a movimentação
normal entre strings, e é um recurso de grande ajuda quando o editor é
utilizado para programação. 

Após instalado o plugin, os seguintes atalhos ficam disponíveis:
\begin{description}
 \item [,w] Movimenta para a próxima posição {\em camel} dentro da string
 \item [,b] Movimenta para a posição {\em camel} anterior dentro da string
 \item [,e] Movimenta para o caractere anterior à próxima posição {\em camel} dentro da string
\end{description}

Fonte: \href{http://eustaquiorangel.com/posts/522}{Blog do Eustáquio Rangel}

\section{Plugin FuzzyFinder}\label{sec:Plugin FuzzyFinder}                                       
Este plugin é a implementação de um recurso do editor 
Texmate\footnote{Editor de textos da Apple com muitos recursos}.
Sua proposta é acessar de forma rápida:

\begin{enumerate}
\item Arquivos \verb|:FuzzyFinderFile|
\item Arquivos recém editados \verb|:FuzzyFinderMruFile|
\item Comandos recém utilizados \verb|:FuzzyFinderMruCmd|
\item Favoritos \verb|:FuzzyFinderAddBookmark, :FuzzyFinderBookmarks|
\item Navegação por diretórios \verb|:FuzzyFinderDir|
\item Tags {\tt :FuzzyFinderTag}
\end{enumerate}

Para ver o plugin em ação acesse este
\href{http://vimeo.com/2938498}{video} para obte-lo acesse 
 \href{http://www.vim.org/scripts/script.php?script\_id=1984}{este endereço},
para instalá-lo basta copiar para o diretório 
{\tt ~/.vim/plugin}.

\section{O plugin EasyGrep}
\label{sec:O plugin EasyGrep}

Usuários de sistemas {\em Unix Like}\footnote{Sistemas da família Unix tipo o
GNU/Linux}, já conhecem o poder do comando {\tt grep}, usando este comando
procuramos palavras dentro de arquivos, este plugin simplifica esta tarefa,
além de permitir a utilização da versão do {\tt grep} nativa do Vim {\tt
vimgrep}, assim usuários do Windows também podem usar este recurso. Um comando
{\tt grep} funciona mais ou menos assim:

\begin{verbatim}
     grep [opções] "padrão" /caminho
\end{verbatim}

Mas no caso do plugin {\em EasyGrep} fica assim:

\begin{verbatim}
     :Grep foo  ........ procura pela palavra 'foo'
     :GrepOptions ...... exibe as opções de uso do plugin
\end{verbatim}

O plugin pode ser obtido no seguinte
\href{http://www.vim.org/scripts/script.php?script\_id=2438#0.9}{endereço}, já sua instalação
é simples, basta copiar o arquivo obtido no link acima para a pasta:
   
\begin{verbatim}
     ~/.vim/plugin .......... no caso do linux
     ~/vimfiles/plugin ...... no caso do windows
\end{verbatim}

Um vídeo de exemplo (na verdade uma animação gif) pode ser visto
\href{http://downloads.veryspeedy.net/vim/EasyGrep.gif}{aqui}.

\section{O plugin {\em SearchComplete}}
Para que o vim complete opções de busca ``com a tecla <tab>'', digita-se uma
palavra parcialmente e o plugin atua, exibindo palavras que tem 
o mesmo início, por exemplo:

\begin{verbatim}
     /merca<tab>
     /mercado
     /mercantil
     /mercadológico
\end{verbatim}

Cada vez que se pressiona a tecla {\tt <tab>} o cursor saltará para 
a próxima ocorrência daquele fragmento de palavra.
Pode-se obter o plugin {\em SearchComplete} no seguinte 
\href{http://www.vim.org/scripts/script.php?script\_id=474}{endereço}, 
e para instalá-lo basta copiá-lo para a pasta apropriada:
    
\begin{verbatim}
     ~/vimfiles/plugin .......... no windows
     ~/.vim/plugin .............. no Gnu/Linux
\end{verbatim}

Há outro plugin similar chamado {\tt CmdlineComplete} disponível  
\href{http://www.vim.org/scripts/script.php?script\_id=2222}{neste link}.


\section{O plugin {\em AutoComplete}}\label{sec:O Plugin AutoComplete}
Este plugin trabalha exibindo sugestões no modo de inserção, à
medida que o usuário digita aparece um {\em popup} com sugestões para possíveis
complementos, bastando pressionar {\tt <Enter>} para aceitar as sugestões.
Neste \href{http://www.vim.org/scripts/script.php?script\_id=1879}{link}, você pode
fazer o {\em download} do plugin.

\section{O plugin {\em Ctags}}
\label{sec:O Plugin Ctags}

{\em Ctags} em si é um programa externo que indexa arquivos de código fonte. Ele lê
o código fonte em busca de identificadores, declarações de função, variáveis, e constrói seu 
índex de referências cruzadas. Mas vamos ao plugin, mesmo por que não estamos no CtagsBook.

Primeiro precisamos ter o arquivos de tags. Para tal, usamos o comando:

\begin{verbatim}
    ctags -R <arquivos>
\end{verbatim}

Normalmente o parâmetro <arquvos> pode ser uma expressão regular do tipo *.[ch]
e afins. Depois de obter o arquivo de tags, você já pode sair usando os atalhos
do plugin para navegar pelo código fonte.  Com o cursor em cima de um
identificador, usando o atalho `\verb|ctrl+]|' o cursor pula diretamente para a
sua declaração. O atalho `\verb|ctrl+o|' volta o cursor para a posição inicial.

{\Large {\ding{45}}} Quando navegando por um código
fonte muito extenso com vários diretórios, é mapear o caminho dos arquivos
usando o caminho absoluto deles no seu diretório de trabalho deste jeito:

\begin{verbatim}
     find $(pwd) -regex ".*py$" | xargs ctags
\end{verbatim}

Assim você pode copiar o arquivo de tags para todos os diretórios e mesmo assim conseguir usar 
os atalhos do plugin para navegar no código fonte.

Pode-se obter o programa {\em Ctags} neste
\href{http://ctags.sourceforge.net/}{link}. O plugin de {\em Ctags} para o Vim está neste 
\href{http://vim.sourceforge.net/scripts/script.php?script\_id=12}{link}'', 
e para instalá-lo basta copiá-lo para a pasta apropriada:
    
\begin{verbatim}
     ~/vimfiles/plugin .......... no windows
     ~/.vim/plugin .............. no Gnu/Linux
\end{verbatim}


\section{O Plugin {\em Project}}

O plugin project acessível através deste \href{http://www.vim.org/scripts/script.php?script\_id=69}{link}
cria toda uma estrutura de gerenciamento de projetos, para programadores é uma funcionalidade
extremamente necessária, costuma-se trabalhar com vários arquivos da mesma família ``extensão'', 
e ao clicar em um dos arquivos do projeto o mesmo é aberto instantaneamente.

\begin{verbatim}
     :Project ......... abre uma janela lateral para o projeto
     \C ............... inicia a criação de um projeto (recursivamente)
     \c ............... inicia a criação de um projeto na pasta local
\end{verbatim}

Após digitar o atalho de criação do projeto aparecerá uma janela para designar um nome
para o mesmo, em seguida digita-se o caminho para o diretório do projeto, após isto
digita-se `.' (ponto) como parâmetro, cria-se um filtro como `{\tt *.py}'.
Para criar uma entrada (acesso ao plugin) no menu do Gvim colocamos
a seguinte linha no `{\tt vimrc}'.

\begin{verbatim}
     amenu &Projetos.togle <Plug>ToggleProject<cr>
\end{verbatim}

Pode-se definir um projeto manualmente assim:

\begin{verbatim}
     nome=~/docs/ CD=. filter="*.txt" {

     }
\end{verbatim}

Ao recarregar o Vim pode-se abrir o {\em Plugin} ``{\tt :Projetc}'' e 
pressionar o atalho `\verb+\r+' para que o mesmo gere um índice dos arquivos
contidos no caminho indicado.
   

\section{O plugin pydiction}
\label{sec:O plugin pydiction}

Um que completa códigos python assim:

\begin{verbatim}
    import sys
    sys.<tab>
\end{verbatim}

O plugin contém dois arquivos:

\begin{enumerate}
    \item {\tt pydiction.py} deve ser colocado no {\tt path}
    \item {\tt pydiction} deve ser colocado em um lugar de sua preferência
\end{enumerate}

Deve-se adicionar algumas linhas do {\tt .vimrc}.
No exemplo abaixo o dicionário é adicionado ao diretório {\tt ~/.vim/dict}
\begin{verbatim}
if has("autocmd")
    autocmd FileType python set complete+=k/.vim/dict/pydiction isk+=.,(
endif " has("autocmd") 
\end{verbatim}

Pode-se obter o plugin  \href{http://www.vim.org/scripts/script.php?script\_id=850}{neste link}.

\section{O plugin FindMate}
\label{sec:O plugin FindMate}

Um plugin que agiliza a busca por arquivos na pasta pessoal, disponível neste
 \href{http://snipt.net/voyeg3r/findmate-plugin-for-vim/}{link}. Basta
coloca-lo na pasta ``{\tt /home/usuario/.vim/plugins/}'' e digitar duas vezes
vírgula e ele substituirá para:

\begin{verbatim}
    :FindMate 
\end{verbatim}

Digita-se então uma palavra e <Enter> para se obter a lista de arquivos que
correspondem ao padrão.

\section{Atualizando a documentação dos plugins}
\label{sec:Atualizando a documentação dos plugins}
\vimhelp{helptags}

Caso seja adicionado algum arquivo de documentação em em `\verb+~/.vim/doc+' pode-se
gerar novamente as tags ``links'' para navegação de ajuda.

\begin{verbatim}
   :helptags ~/.vim/doc
\end{verbatim}

\chapter{Buscas e Substituições}\label{Buscas e substituições}

Para fazer uma busca, certifique-se de que está em modo normal,
pressione ``/'' e digite a expressão a ser procurada. \\


Para encontrar a primeira ocorrência de ``foo'' no texto:

\begin{verbatim}
     /foo
\end{verbatim}

Para repetir a busca basta pressionar a tecla ``\verb+n+'' e para
repetir a busca em sentido oposto ``\verb+N+''.

\begin{verbatim}
     /teste/+3
\end{verbatim}

Posiciona o cursor três linhas após a ocorrência da palavra ``teste'' \\


\section{Usando ``Expressões Regulares'' em buscas}

\begin{verbatim}
     / ........... inicia uma busca (modo normal)
     \%x69 ....... código da letra `i'
     /\%x69 ...... localiza a letra `i' - hexadecimal 069
     
     \d .......... localiza números
     ^ ........... começo de linha
     $ ........... final de linha
     \+ .......... um ou mais
     /^\d\+$ ..... localiza somente dígitos
     
     \s .......... localiza espaços
     /\s\+$ ...... localiza espaços no final da linha 
\end{verbatim}

Para aprender mais sobre Expressões Regulares leia:

\begin{enumerate}
  \item \url{http://guia-er.sourceforge.net}
  \item :help regex
\end{enumerate}

Um meio mais rápido para encontrar a próxima ocorrência de uma palavra sob o
cursor é teclar \verb|*|. Para encontrar uma ocorrência anterior da palavra
sob o cursor, tecle \verb|#|---em ambos os casos o cursor deve estar
posicionado sobre a palavra que deseja procurar.

\section{Destacando padrões}\label{sec:Destacando padrões}

Você pode destacar linhas com mais de 30 caracteres assim:

\begin{verbatim}
     :match ErrorMsg /\%>30v/ . destaca linhas maiores que 30 caracteres
     :match none .............. remove o destaque
     para saber mais leia :h %>
\end{verbatim}


\section{Editando em nova janela}\label{Editando em nova janela}

Caso deseje manter o arquivo atual e editar `simultaneamente' outro arquivo
pode dividir a janela assim:

\begin{verbatim}
     Ctrl-w-n
\end{verbatim}

Para mais detalhes sobre janelas acesse o capítulo
\ref{cha:Trabalhando com janelas} na página \pageref{cha:Trabalhando
com janelas}.


\section{Inserindo linha antes e depois}

Suponha que se queira um comando, considere \verb|,t|, que faça com que a
linha indentada corrente passe a ter uma linha em branco antes e depois; isto
pode ser obtido pelo seguinte mapeamento:

\begin{verbatim}
     :map ,t <Esc>:.s/^\(\s\+\)\(.*\)/\r\1\2\r/g<cr>
\end{verbatim}
     
Explicando:
     
\begin{verbatim}
     : ................ entra no modo de comando
     map ,t ........... mapeia ,t para a função desejada
     <Esc> ............ ao executar sai do modo de inserção
     s/isto/aquilo/g .. substitui isto por aquilo
     : ................ inicia o modo de comando
     . ................ na linha corrente
     s ................ substitua
     ^ ................ começo de linha
     \s\+ ............. um espaço ou mais (barras são escapes)
     .* ............... qualquer coisa depois
     \(grupo\) ........ agrupo para referenciar com \1
     \1 ............... repete na substituição o grupo 1
     \r ............... insere uma quebra de linha
     g ................ em todas as ocorrências da linha
     <cr> ............. Enter
\end{verbatim}

\section{Obtendo informações do arquivo}

\begin{verbatim}
     ga ............. mostra o código do caractere em decimal hexa e octal
     ^g ............. mostra o caminho e o nome do arquivo
\end{verbatim}

Obs: O código do caractere pode ser usado para substituições,
especialmente em se tratando de caracteres de controle como tabulações
``\verb|^I|'' ou final de linha DOS/Windows ``\verb|\%x0d|''. Você pode apagar os
caracteres de final de linha Dos/Windows usando uma simples
substituição, veja mais adiante:

\begin{verbatim}
     :%s/\%x0d//g
\end{verbatim}

Na seção \ref{cha:Como editar preferências no Vim} há um código para a barra de
status que faz com que a mesma exiba o código do caractere sob o cursor na
seção \ref{Funçã para barra de status}.

\section{Trabalhando com registradores}
\label{Trabalhando com registradores}

Você não precisa copiar e colar diferentes partes do texto para uma
mesma área de transferência.  Para isso, você pode usar os
``registros''.  Os registradores são indicados por aspas seguido por uma letra.
Exemplos: ``a'', ``b'', ``c'', etc.

Como copiar o texto para um registrador? É simples: basta especificar
o nome do registrador antes:

\begin{verbatim}
     "add ... apaga uma linha, copiando seu conteúdo para o registrador a
     "bdd ... apaga uma linha, copiando seu conteúdo para o registrador b
     "ap .... cola" o conteúdo do registrador a
     "ab .... cola" o conteúdo do registrador b
     "x3dd .. apaga 3 linhas, copiando o conteúdo para o registrador ``x''
     "ayy  .. copia uma linha, sem apagar, para o registrador a
     "a3yy .. copia 3 linhas, sem apagar, para o registrador a
     "ayw  .. copia uma palavra, sem apagar, para o registrador a
     "a3yw .. copia 3 palavras, sem apagar, para o registrador a
\end{verbatim}

No ``modo de inserção'', como visto anteriormente, você pode usar um atalho
para colar rapidamente o conteúdo de um registrador.

\begin{verbatim}
     Ctrl-r (registro)
\end{verbatim}

Para colar o conteúdo do registrador ``a''

\begin{verbatim}
     Ctrl-r a
\end{verbatim}

Para copiar a linha atual para a área de transferência

\begin{verbatim}
     "+yy
\end{verbatim}

Para colar da área de transferência

\begin{verbatim}
     "+p
\end{verbatim}

\section{Edições complexas }
\label{Edições complexas }

Trocando palavras de lugar: coloque o cursor no espaço antes da 1ª palavra e digite:

\begin{verbatim}
     deep
\end{verbatim}

Trocando letras de lugar:

\begin{verbatim}
     xp
\end{verbatim}

Trocando linhas de lugar:

\begin{verbatim}
     ddp
\end{verbatim}

Tornando todo o texto maiúsculo
 gggUG

\section{Indentando }

\begin{verbatim}
     >> ..... Indenta a linha atual
     ^T ..... Indenta a linha atual em modo de inserção
     ^D ..... Remove indentação em modo de inserção
     >ip .... indenta o parágrafo atual
\end{verbatim}

\section{Corrigindo a indentação de códigos}
\label{Corrigindo a indentação de códigos}
Selecione o bloco de código, por exemplo

\begin{verbatim}
     vip  ..... visual ``h'' (selecione este parágrafo)
     =  ....... corrija a indentação do que selecionei :)
\end{verbatim}

\section{Usando o file explorer}
\label{Usando o file explorer}
O Vim navega na árvore de diretórios com o comando

\begin{verbatim}
     vim .
\end{verbatim}

Use o ``j'' para descer e o ``k'' para subir ou Enter para editar o
arquivo selecionado. Outra dica é pressionar F1 ao abrir o
FileExplorer do Vim, você encontra dicas adicionais sobre este modo de
operação do Vim.

\section{Selecionando ou deletando conteúdo de tags html}
\label{Selecionando ou deletando conteúdo de tags html}

\begin{verbatim}
     <tag> conteúdo da tag </tag>
     basta usar (em modo normal) as teclas
     vit ............... visual ``inner tag | esta tag''
\end{verbatim}

Este recurso também funciona com parênteses

\begin{verbatim}
     vi( ..... visual select
     vi" ..... visual select
     di( ..... delete inner (, ou seja, seu conteúdo
\end{verbatim}


\section{Substituições }
\label{Substituições }

Para fazer uma busca, certifique-se de que está em modo normal, em
seguida digite use o comando ``s'', conforme será explicado.

Para substituir ``foo'' por ``bar'' na linha atual:

\begin{verbatim}
     :s/foo/bar
\end{verbatim}

Para substituir ``o'' por ``r'' da primeira à décima linha do arquivo:

\begin{verbatim}
     :1,10 s/foo/bar
\end{verbatim}

Para substituir ``foo'' por ``bar'' da primeira à última linha do arquivo:

\begin{verbatim}
     :1,$ s/foo/bar
\end{verbatim}

Ou simplesmente:

\begin{verbatim}
     :% s/foo/bar
\end{verbatim}

\begin{verbatim}
     $ ... significa para o Vim final do arquivo
     % ... representa o arquivo atual
\end{verbatim}

O comando ``s'' possui muitas opções que modificam seu comportamento.

\section{Exemplos }
\label{Exemplos }

Busca usando alternativas:

\begin{verbatim}
     /end\(if\|while\|for\)
\end{verbatim}

Buscará ``if'', ``while'' e ``for''.  Observe que é necessário `escapar' os
caracteres \verb|\(|, \verb|\|| e \verb|\)|, caso contrário eles serão
interpretados como caracteres comuns.

Quebra de linha

\begin{verbatim}
     /quebra\nde linha
\end{verbatim}

Ignorando maiúsculas e minúsculas

\begin{verbatim}
     /\cpalavra
\end{verbatim}

Usando \verb|\c| o Vim encontrará ``{\em{palavra}}'', ``{\em{Palavraa}}'' ou
até mesmo ``{\em{PALAVRA}}''. Uma dica é colocar no seu arquivo de
configuração ``vimrc'' veja o capítulo \ref{cha:Como editar preferências no Vim}.

\begin{verbatim}
     set ignorecase .. ignora maiúsculas e minúsculas na bucsca
     set smartcase ... se busca contiver maiúsculas ele passa a considerá-las
     set hlsearch .... mostra o que está sendo buscado em cores
     set incsearch ... ativa a busca incremental
\end{verbatim}

se você não sabe ainda como colocar estas preferências no arquivo de configuração pode
ativa-las em modo de comando precedendo-as com dois pontos, assim:

\begin{verbatim}
     :set ignorecase<Enter>
\end{verbatim}

Procurando palavras repetidas

\begin{verbatim}
     /\<\(\w*\) \1\>
\end{verbatim}

Multilinha

\begin{verbatim}
     /Hello\_s\+World
\end{verbatim}

Buscará `World', separado por qualquer número de espaços,
incluindo quebras de linha. Buscará as três seqüências:

\begin{verbatim}
     Hello World
     
     Hello    World
     
     Hello
     World
\end{verbatim}

Buscar linhas de até 30 caracteres de comprimento

\begin{verbatim}
     /^.\{,30\}$
\end{verbatim}

\begin{verbatim}
     ^ ...... representa começo de linha
\end{verbatim}

Apaga todas as tags html/xml de um arquivo

\begin{verbatim}
     :%s/<[^>]*>//g
\end{verbatim}

Apaga linhas vazias

\begin{verbatim}
     :%g/^$/d
\end{verbatim}

Ou

\begin{verbatim}
     :%s/^[\ \t]*\n//g
\end{verbatim}

Remover duas ou mais linhas vazias entre parágrafos diminuindo para
uma só linha vazia.

\begin{verbatim}
     :%s/\(^\n\{2,}\)/\r/g
\end{verbatim}

Você pode criar um mapeamento e colocar no seu ~/.vimrc

\begin{verbatim}
     map ,s <Esc>:%s/\(^\n\{2,}\)/\r/g<cr>
\end{verbatim}

No exemplo acima, ``,s'' é um mapeamento para reduzir linhas em branco
sucessivas para uma só  \\


Remove não dígitos (não pega números)

\begin{verbatim}
     :%s/^\D.*//g
\end{verbatim}

Remove final de linha DOS/Windows \verb|^M| que tem código hexadecimal igual a
``0d''

\begin{verbatim}
     :%s/\%x0d//g
\end{verbatim}

Troca palavras de lugar usando expressões regulares

\begin{verbatim}
     :%s/\(.\+\)\s\(.\+\)/\2 \1/
\end{verbatim}

Modificando todas as tags html para minúsculo

\begin{verbatim}
     :%s/<\([^>]*\)>/<\L\1>/g
\end{verbatim}

Move linhas 10 a 12 para além da linha 30

\begin{verbatim}
     :10,12m30
\end{verbatim}

\section{O comando global ``g''}\label{sec:O comando global ``g''}

buscando um padrão e gravando em outro arquivo

\begin{verbatim}
     :'a,'b g/^Error/ . w >> errors.txt
\end{verbatim}

Apenas imprimir linhas que contém determinada palavra, isto é útil 
quando você quer ter uma visão sobre um determina aspecto 
do seu arquivo vejamos:

\begin{verbatim}
     :set nu ..... habilita numeração 
     :g/Error/p .. apenas mostra as linhas correspondentes
\end{verbatim}

numerar linhas

\begin{verbatim}
     :let i=1 | g/^/s//\=i."\t"/ | let i=i+1
\end{verbatim}

Para copiar linhas começadas com {\em Error} para o final do arquivo faça:

\begin{verbatim}
     :g/^Error/ copy $
\end{verbatim}

Obs: O comando {\em copy} pode ser abreviado `co' ou ainda você pode usar `t'
para mais detalhes leia

\begin{verbatim}
     :h co
\end{verbatim}

Entre as linhas que contiverem ``fred'' e ``joe'' substitua

\begin{verbatim}
     :g/fred/,/joe/s/isto/aquilo/gic
\end{verbatim}

As opções `gic' correspondem a {\em global}, {\em ignore case} e {\em
confirm}, podendo ser omitidas deixando só o {\em global}. \\


Pegar caracteres numéricos e jogar no final do arquivo?

\begin{verbatim}
     :g/^\d\+.*/m $
\end{verbatim}

Inverter a ordem das linhas do arquivo?

\begin{verbatim}
     :g/^/m0
\end{verbatim}

Apagar as linhas que contém {\em Line commented}

\begin{verbatim}
     :g/Line commented/d
\end{verbatim}

Copiar determinado padrão para um registro

\begin{verbatim}
     :g/pattern/ normal "Ayy
\end{verbatim}

Copiar linhas que contém um padrão e a linha subsequente para o final

\begin{verbatim}
     :g/padrão/;+1 copy $
\end{verbatim}

Deletar linhas que não contenham um padrão

\begin{verbatim}
     :v/dicas/d  ..... deleta linhas que não contenham `dicas'
\end{verbatim}

Incrementar números no começo da linha

\begin{verbatim}
     :.,20g/^\d/exe "normal! \<c-a>"
\end{verbatim}

\section{Dicas }
Para colocar a última busca em uma substituição faça:

\begin{verbatim}
     :%s/Ctrl-r//novo/g
\end{verbatim}

A dupla barra corresponde ao ultimo padrão procurado, e portanto o
comando abaixo fará a substituição da ultima busca por casinha

\begin{verbatim}
     :%s//casinha/g
\end{verbatim}

\section{Filtrando arquivos com o vimgrep}
\label{Filtrando arquivos com o vimgrep}

Por vezes sabemos que aquela anotação foi feita, mas no momento esquecemos em qual
arquivo está, no exemplo abaixo procuramos a palavra dicas à partir da nossa pasta pessoal
pela palavra `dicas' em todos os arquivos com extensão `txt'.

\begin{verbatim}
     ~/ ............ equivale a /home/user
     :lvimgrep /dicas/ ~/**/*.txt | ls
\end{verbatim}


\section{Copiar a partir de um ponto}

\begin{verbatim}
     :19;+3 co $
\end{verbatim}

O Vim sempre necessita de um intervalo (inicial e final) mas se você
usar ``;'' ele considera a primeira linha como segundo ponto do
intervalo, e no caso acima estamos dizendo (nas entrelinhas) linhas
19 e 19+3     \\


De forma análoga podemos usar como referência um padrão qualquer

\begin{verbatim}
     :/palavra/;+10 m 0
\end{verbatim}

O comando acima diz: à partir da linha que contém ``palavra'' incluindo as 10 próximas linhas
mova ``m'' para a primeira linha ``0'', ou seja, antes da linha 1.

\section{Dicas das lista vi-br}

 Fonte: \url{http://groups.yahoo.com/group/vi-br/message/853}

 Problema:
 Essa deve ser uma pergunta comum.
 Suponha o seguinte conteúdo de arquivo:

\begin{verbatim}
     ... // várias linhas
     texto1000texto    // linha i
     texto1000texto    // linha i+1
     texto1000texto    // linha i+2
     texto1000texto    // linha i+3
     texto1000texto    // linha i+4
     ... // várias linhas
\end{verbatim}

Gostaria de um comando que mudasse para

\begin{verbatim}
     ... // várias linhas
     texto1001texto    // linha i
     texto1002texto    // linha i+1
     texto1003texto    // linha i+2
     texto1004texto    // linha i+3
     texto1005texto    // linha i+4
     ... // várias linhas
\end{verbatim}

 Ou seja, somasse 1 a cada um dos números entre os textos
 especificando como range as linhas i,i+4

\begin{verbatim}
     :10,20! awk 'BEGIN{i=1}{if (match($0, ``+'')) print ``o''
     (substr($0, RSTART, RLENGTH) + i++) ``o'``}''
\end{verbatim}

 Mas muitos sistemas não tem awk, e logo a melhor solução mesmo é usar o Vim:

\begin{verbatim}
     :let i=1 | 10,20 g/texto\d\+texto/s/\d\+/\=submatch(0)+i/ | let i=i+1
\end{verbatim}

Observação: 10,20 é o intervalo, ou seja, da linha 10 até a linha 20

\begin{verbatim}
     :help /
     :help :s
     :help pattern
\end{verbatim}

\section{Dicas do dicas-l}

fonte: \url{http://www.dicas-l.com.br/dicas-l/20081228.php}

\section{Junção de linhas com Vim}
\label{Junção de linhas com Vim}
Colaboração: Rubens Queiroz de Almeida

Recentemente precisei combinar, em um arquivo, duas linhas
consecutivas. O arquivo original continha linhas como:

\begin{verbatim}
     Matrícula: 123456
     Senha: yatVind7kned
     Matrícula: 123456
     Senha: invanBabnit3
\end{verbatim}

E assim por diante. Eu precisava converter este arquivo para algo como:

\begin{verbatim}
     Matrícula: 123456 - Senha: yatVind7kned
     Matrícula: 123456 - Senha: invanBabnit3
\end{verbatim}

Para isto, basta executar o comando:

\begin{verbatim}
     :g/^Matrícula/s/\n/ - /
\end{verbatim}

Explicando:

\begin{verbatim}
     s/isto/aquilo/g .. substitui isto por aquilo
     g ................ comando global
     /................. inicia padrão de busca
     ^ ................ indica começo de linha
     Matrícula ........ palavra a ser buscada
     s ................ inicia substituição
     /\n/ - / ......... troca quebra de linha (\n), por -
\end{verbatim}

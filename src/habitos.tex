\chapter{Hábitos para Edição Efetiva}
\label{cha:Hábitos para edição efetiva}


Um dos grandes problemas relacionados com os softwares é sua subutilização. Por
inércia o usuário tende a aprender o mínimo para a utilização de um programa e
deixa de lado recursos que poderiam lhe ser de grande valia. O mantenedor do
Vim  Bram Moolenaar\footnote{http://www.moolenaar.net} recentemente publicou vídeos
e manuais sobre os ``7 hábitos para edição efetiva de
textos\footnote{http://br-linux.org/linux/7-habitos-da-edicao-de-texto-efetiva}'',
este capítulo pretende resumir alguns conceitos mostrados por Bram Moolenaar em
seu artigo.

\section{Mova-se rapidamente no texto}
\label{sec:Mova-se rapidamente no texto}

Leia sobre como mover-se no documento no capítulo \ref{cha:Movendo-se no Documento}


\section{Use marcas}
veja a seção \ref{sec:Usando marcas} na página \pageref{sec:Usando marcas}.

\begin{verbatim}
     ma ..... em modo normal cria uma marca `a'
     'a ..... move o cursor até a marca `a'
     d'a .... deleta até a marca `a'
     y'a .... copia até a marca `a'
\end{verbatim}



\begin{verbatim}
     gg ... vai para a linha 1 do arquivo
     G .... vai para a última linha do arquivo
     0 .... vai para o início da linha
     $ .... vai para o fim da linha
     fx ... pula até a próxima ocorrência de ``x''
     dfx .. deleta até a próxima ocorrência de ``x''
     g, ... avança na lista de alterações
     g; ... retrocede na lista de alterações
     p .... cola o que foi deletado/copiado abaixo
     P .... cola o que foi deletado/copiado acima
     H .... posiciona o cursor no primeiro caractere da tela
     M .... posiciona o cursor no meio da tela
     L .... posiciona o cursor na última linha da tela
\end{verbatim}

\begin{verbatim}
     * Use asterisco  para localizar a palavra sob o cursor
     * Use o percent % serve para localizar fechamento de parêntese chaves etc
\end{verbatim}

\begin{verbatim}
     '.  apostrofo + ponto retorna ao último local editado
     '' retorna ao local do ultimo salto
\end{verbatim}

Suponha que você está procurando a palavra `argc':

\begin{verbatim}
     /argc
\end{verbatim}

Digita `n' para buscar a próxima ocorrência

\begin{verbatim}
     n
\end{verbatim}

Um jeito mais fácil seria:

\begin{verbatim}
     "coloque a linha abaixo no seu vimrc
     :set hlsearch
\end{verbatim}

Agora use asterisco para destacar todas as ocorrências do padrão desejado
e use a letra `n' para pular entre ocorrências, caso deseje seguir o caminho
inverso use `N'.

\section{Use quantificadores}
\label{Use quantificadores}
Em modo normal você pode fazer

\begin{verbatim}
     10j ..... desce 10 linhas
     5dd ..... apaga as próximas 5 linhas
     :50 ..... vai para a linha 50
     50gg .... vai para a linha 50
\end{verbatim}


\section{Edite vários arquivos de uma só vez }
\label{Edite vários arquivos de uma só vez }

O Vim pode abrir vários arquivos que contenham um determinado padrão.
Um exemplo seria abrir dezenas de arquivos html e trocar a ocorrência
bgcolor=``f'' Para bgcolor=``e'' Usaríamos o seguinte comando

\begin{verbatim}
     vim *.html :bufdo :%s/bgcolor=``f''/bgcolor=``e''/g :wall :qall
\end{verbatim}

Ainda com relação à edição de vários arquivos poderia-mos abrir alguns
arquivos txt e mudar de um para o outro assim:

\begin{verbatim}
     :wn
\end{verbatim}

O ``w'' significa gravar e o ``n'' significa {\em next}, ou seja, gravaria-mos
o que foi modificado no arquivo atual e mudaríamos para o próximo.

Veja também: \ref{cha:Movendo-se no documento}

\section{Não digite duas vezes}
\label{Não digite duas vezes}

\begin{itemize}
\item O Vim complementa com ``tab''. Veja mais na seção \ref{Complementação com ``tab''} na página \pageref{Complementação com ``tab''}.
\item Use macros. Detalhes na seção \ref{Macros: gravando comandos}
página \pageref{Macros: gravando comandos}.
\item Use abreviações coloque abreviações como abaixo em seu \verb|~/.vimrc|. Veja mais na seção \ref{Abreviações}.
\item as abreviações fazem o mesmo que auto-correção e auto-texto em outros editores
\end{itemize}

\begin{verbatim}
     iab tambem também
     iab linux GNU/Linux
\end{verbatim}



* No modo de inserção você pode usar

\begin{verbatim}
     Ctrl-y  ....... copia caractere a caractere a linha acima
     Ctrl-e  ....... copia caractere a caractere a linha abaixo
     Ctrl-x Ctrl-l .. completa linhas inteiras
\end{verbatim}

* Para um trecho muito copiado coloque o seu conteúdo em um registrador

\begin{verbatim}
     "ayy ... copia a linha atual para o registrador ``a''
     "ap  ... cola o registrador ``a''
\end{verbatim}

Crie abreviações para erros comuns no seu arquivo de configuração (~/.vimrc)

\begin{verbatim}
     iabbrev teh the
     syntax keyword WordError teh
\end{verbatim}

As linhas acima criam uma abreviação para erro de digitação da palavra `the'
e destaca textos que você abrir que contenham este erro.

\section{Use dobras}\label{sec:Use folders}

O Vim pode ocultar partes do texto que não estão sendo utilizadas permitindo
uma melhor visualização do conteúdo. Mais detalhes no capítulo
\ref{cha:Folders} página \pageref{cha:Folders}.

\section{Use autocomandos}
\label{Use autocomandos}

No arquivo de configuração do Vim \verb|~/.vimrc| você pode criar comandos
automáticos que serão executados diante de uma determinada
circunstância

O comando abaixo será executado em qualquer arquivo existente, posicionando o cursor no último local editado

\begin{verbatim}
     "autocmd BufEnter * lcd %:p:h
     autocmd BufReadPost *
       \ if line("'\"") > 0 && line("'\"") <= line("$") |
       \   exe "normal g`\"" |
       \ endif
\end{verbatim}


Grupo de comandos para arquivos do tipo ``html''. Observe que o
autocomando carrega um arquivo de configuração do Vim exclusivo para o
tipo html/htm e no caso de arquivos novos ``BufNewFile'' ele já cria um
esqueleto puxando do endereço indicado.

\begin{verbatim}
     augroup html
      au! <--> Remove all html autocommands
       au!
       au BufNewFile,BufRead *.html,*.shtml,*.htm set ft=html
       au BufNewFile,BufRead,BufEnter  *.html,*.shtml,*.htm so ~/docs/vim/.vimrc-html
       au BufNewFile *.html 0r ~/docs/vim/skel.html
       au BufNewFile *.html*.shtml,*.htm /body/+  " coloca o cursor após o corpo <body>
       au BufNewFile,BufRead *.html,*.shtml,*.htm set noautoindent
     augroup end
\end{verbatim}

Documentação sobre autocomandos do Vim \url{http://www.vim.org/htmldoc/autocmd.html}.

\section{Use o file explorer}\label{Use o file explorer}

O Vim pode navegar em pastas assim:

\begin{verbatim}
     vim .
\end{verbatim}

Você pode usar ``j'' e ``k'' para navegar e {\tt Enter} para editar o arquivo
selecionado

\section{Torne as boas práticas um hábito }\label{Torne as boas práticas um hábito }

Para cada prática produtiva procure adquirir um hábito e mantenha-se
atento ao que pode ser melhorado. Imagine tarefas complexas, procure
um meio melhor de fazer e torne um hábito.

\section{Referências}
\label{Referências}
\begin{itemize}
   \item \url{http://www.moolenaar.net/habits\_2007.pdf} por Bram Moolenaar
   \item \url{http://vim.wikia.com/wiki/Did\_you\_know}
\end{itemize}

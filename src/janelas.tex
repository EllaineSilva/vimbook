%%%%%%%%%%%%%%%%%%%%%%%%%%%%%%%%%%%%%%%%%%%%%%%%%%%%%%%%%%%%%%%%%%%%%%%%
% vim:enc=utf-8:ts=5:sw=5:et:ff=unix:
%%%%%%%%%%%%%%%%%%%%%%%%%%%%%%%%%%%%%%%%%%%%%%%%%%%%%%%%%%%%%%%%%%%%%%%%

\chapter{Trabalhando com Janelas}\label{cha:Trabalhando com janelas}

O Vim trabalha com o conceito de múltiplos ``buffers''. Cada
``buffer'' é um arquivo carregado para edição. Um ``buffer'' pode
estar visível ou não, e é possível dividir a tela em janelas, de forma
a visualizar mais de um ``buffer'' simultaneamente. 

\section{Modos de divisão da janela}

Como foi dito acima, é possível visualizar mais de um buffer ao mesmo 
tempo, e isso pode ser feito utilizando {\em tab} ou {\em split}.


\subsection{Utilizando abas {\em tab}}

A partir do Vim 7 foi disponibilizada a função de abrir arquivos em
abas, portanto é possível ter vários ``buffers'' abertos em abas distintas
e alternar entre elas facilmente. Os comandos para utilização das abas
são:

\begin{verbatim}
     :tabnew ........... Abre uma nova tab
     :tabprevious ...... Vai para a tab anterior
     :tabnext .......... Vai para a próxima tab
\end{verbatim}


\subsection{Utilizando split horizontal}

Enquanto os comandos referentes a {\em tab} deixam a janela inteira disponível
para o texto e apenas cria uma pequena aba na parte superior, o comando {\em split} 
literamente divide a tela atual em duas para visualização simultânea
dos ``buffers'' (seja ele o mesmo ou outro diferente).
Esse é o split padrão do Vim mas pode ser alterado facilmente colocando a linha
abaixo no seu \verb|~/.vimrc|:

% deixei a linha de explicação mais curta para não exceder a largura da página
\begin{verbatim}
     :set splitright .... split padrão para vertical
\end{verbatim}


\subsection{Utilizando split vertical}

O split vertical funciona da mesma maneira que o split horizontal,
sendo a unica diferença o modo como a tela é dividida, pois nesse
caso a tela é dividida verticalmente.


\section{Abrindo e fechando janelas }

\begin{verbatim}
     Ctrl-w-s ........... Divide a janela atual em duas horizontalmente (:split)
     Ctrl-w-v ........... Divide a janela atual em duas verticalmente (:vsplit)
     Ctrl-w-o ........... Faz a janela atual ser a única (:only)
     Ctrl-w-n ........... Abre nova janela (:new)
     Ctrl-w-q ........... Fecha a janela atual (:quit)
     Ctrl-w-c ........... Fecha a janela atual (:close)
\end{verbatim}

{\Large \ding{45}} Lembrando que o Vim considera todos os arquivos como ``buffers''
portanto quando um arquivo é fechado, o que está sendo fechado é a visualização
do mesmo, pois ele continua aberto no ``buffer''.


\section{Salvando e saindo}
É possível salvar todas as janelas facilmente, assim como sair tambem:

\begin{verbatim}
     :wall ............. salva todos `write all'
     :qall ............. fecha todos `quit all'
\end{verbatim}


\section{Manipulando janelas }

\begin{verbatim}
     Ctrl-w-w ......... Alterna entre janelas
     Ctrl-w-j ......... desce uma janela `j'
     Ctrl-w-k ......... sobe  uma janela `k'
     Ctrl-w-l ......... move para a janela da direta `l'
     Ctrl-w-h ......... move para a janela da direta `h'
     Ctrl-w-r ......... Rotaciona janelas na tela
     Ctrl-w-+ ......... Aumenta o espaço da janela atual
     Ctrl-w-- ......... Diminui o espaço da janela atual
\end{verbatim}

\section{File Explorer }
\label{File Explorer }
\vimhelp{buffers windows}
Para abrir o gerenciador de arquivos do Vim use:

\begin{verbatim}
     :Vex ............ abre o file explorer verticalmente
     :Sex ............ abre o file explorer em nova janela
     :Tex ............ abre o file explorer em nova aba
     :e .............. abre o file explorer na janela atual
     após abrir chame a ajuda <F1>
\end{verbatim}

Para abrir o arquivo sob o cursor em nova janela coloque a linha abaixo no seu \verb|~/.vimrc|

\begin{verbatim}
     let g:netrw_altv = 1
\end{verbatim}

É possível mapear um atalho ``no caso abaixo F2'' para abrir o File Explorer.

\begin{verbatim}
     map <F2> <Esc>:Vex<cr>
\end{verbatim}


{\Large {\ding{45}}} Ao editar um arquivo no qual há referência a um outro
arquivo, por exemplo: `{\tt /etc/hosts}', pode-se usar o atalho `{\tt Ctrl-w
f}' para abri-lo em nova janela, ou `{\tt gf}' para abri-lo na janela atual.
Mas é importante posicionar o cursor sobre o nome do arquivo.  Veja também
mapeamentos na seção \ref{Mapeamentos} página \pageref{Mapeamentos}.

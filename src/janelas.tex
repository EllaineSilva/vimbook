%%%%%%%%%%%%%%%%%%%%%%%%%%%%%%%%%%%%%%%%%%%%%%%%%%%%%%%%%%%%%%%%%%%%%%%%
% vim:enc=utf-8:ts=5:sw=5:et
%%%%%%%%%%%%%%%%%%%%%%%%%%%%%%%%%%%%%%%%%%%%%%%%%%%%%%%%%%%%%%%%%%%%%%%%

\chapter{Trabalhando com Janelas}\label{cha:Trabalhando com janelas}

O Vim trabalha com o conceito de múltiplos ``buffers''. Cada
``buffer'' é um arquivo carregado para edição. Um ``buffer'' pode
estar visível ou não, e é possível dividir a tela em janelas, de forma
a visualizar mais de um ``buffer'' simultaneamente.

\section{Dividindo a janela }
Observação: \verb+Ctrl = ^+

\begin{verbatim}
     Ctrl-w-s   Divide a janela atual em duas (:split)
     Ctrl-w-o   Faz a janela atual ser a única (:only)
     Ctrl-w-n   Abre nova janela (:new)
     Ctrl-w-q   Fecha a janela atual (:quit)
\end{verbatim}

Caso tenha duas janelas e use o atalho acima \verb|^wo|, é recomendado salvar
tudo ao fechar, pois apesar de a outra janela estar fechada o arquivo
ainda estará carregado, portanto faça:

\begin{verbatim}
     :wall ... salva todos `write all'
     :qall ... fecha todos `quite all'
\end{verbatim}

\section{Abrindo e fechando janelas }

\begin{verbatim}
     Ctrl-w-n   Abre uma nova janela acima
     Ctrl-w-q   Fecha a janela atual
     Ctrl-w-c   Fecha a janela atual (:close)
\end{verbatim}

\section{Manipulando janelas }

\begin{verbatim}
     Ctrl-w-w ... Alterna entre janelas
     Ctrl-w-j ... desce uma janela `j'
     Ctrl-w-k ... sobe  uma janela `k'
     Ctrl-w-r ... Rotaciona janelas na tela
     Ctrl-w-+ ... Aumenta o espaço da janela atual
     Ctrl-w-- ... Diminui o espaço da janela atual
\end{verbatim}

\section{File Explorer }
\label{File Explorer }
Para abrir o gerenciador de arquivos do Vim use:

\begin{verbatim}
     :Vex ........... abre o file explorer verticalmente
     :Sex ........... abre o file explorer em nova janela
     :e .   ......... abre o file explorer na janela atual
     após abrir chame a ajuda <F1>
\end{verbatim}

Para abrir o arquivo sob o cursor em nova janela coloque a linha abaixo no seu \verb|~/.vimrc|

\begin{verbatim}
     let g:netrw_altv = 1
\end{verbatim}

É possível mapear um atalho ``no caso abaixo F2'' para abrir o File Explorer.

\begin{verbatim}
     map <F2> <Esc>:Vex<cr>
\end{verbatim}

Maiores informações:

\begin{verbatim}
     :help buffers
     :help windows
\end{verbatim}

\section{Dicas}
Ao editar um arquivo no qual há referência a um outro arquivo, por exemplo: 
``\verb+/etc/hosts+'', pode-se usar o atalho \verb+Ctrl-w f+ para abri-lo.
Mas é importante posicionar o cursor sobre o nome do arquivo.

Veja também mapeamentos na seção \ref{Mapeamentos} página \pageref{Mapeamentos}.
